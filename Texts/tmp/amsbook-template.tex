%% filename: amsbook-template.tex
%% version: 1.1
%% date: 2014/07/24
%%
%% American Mathematical Society
%% Technical Support
%% Publications Technical Group
%% 201 Charles Street
%% Providence, RI 02904
%% USA
%% tel: (401) 455-4080
%%      (800) 321-4267 (USA and Canada only)
%% fax: (401) 331-3842
%% email: tech-support@ams.org
%% 
%% Copyright 2006, 2008-2010, 2014 American Mathematical Society.
%% 
%% This work may be distributed and/or modified under the
%% conditions of the LaTeX Project Public License, either version 1.3c
%% of this license or (at your option) any later version.
%% The latest version of this license is in
%%   http://www.latex-project.org/lppl.txt
%% and version 1.3c or later is part of all distributions of LaTeX
%% version 2005/12/01 or later.
%% 
%% This work has the LPPL maintenance status `maintained'.
%% 
%% The Current Maintainer of this work is the American Mathematical
%% Society.
%%
%% ====================================================================

%    AMS-LaTeX v.2 driver file template for use with amsbook
%
%    Remove any commented or uncommented macros you do not use.

\documentclass[oneside]{amsbook}

%    For use when working on individual chapters
%\includeonly{}

%    Include referenced packages here.

\usepackage{amsmath, amssymb}
\usepackage{indentfirst}
\usepackage{url, cite, hyperref}
\hypersetup{hidelinks}


\newtheorem{theorem}{Theorem}[chapter]
\newtheorem{lemma}[theorem]{Lemma}

\theoremstyle{definition}
\newtheorem{definition}[theorem]{Definition}
\newtheorem{example}[theorem]{Example}
\newtheorem{xca}[theorem]{Exercise}

\theoremstyle{remark}
\newtheorem{remark}[theorem]{Remark}

\numberwithin{section}{chapter}
\numberwithin{equation}{chapter}

%    For a single index; for multiple indexes, see the manual
%    "Instructions for preparation of papers and monographs:
%    AMS-LaTeX" (instr-l.pdf in the AMS-LaTeX distribution).
% \makeindex

\begin{document}

\frontmatter

\title{Boussinesq Approximation and Nondimensionalization for Mantle Convection}

%    Remove any unused author tags.

%    author one information
\author{Ivan Cherkashin}
\address{University of California, Davis}
\curraddr{}
\email{icherkashin@ucdavis.edu}
% \email{egpuckett@ucdavis.edu}

%    author two information
% \author{Elbridge G. Puckett}
\address{}
\curraddr{}
% \email{egpuckett@ucdavis.edu}
\thanks{}

% \subjclass[2015]{Primary }

\keywords{incompressible Stokes equations, mantle, convection, nondimensionalization, scaling, Boussinesq approximation}

\date{\today}

\begin{abstract}

The simplest from of equations of the Boussinesq approximation is nondimensionalized in the context of numerical modeling of mantle convection. Physical considerations underlying the appropriate choices of time, velocity, and pressure scales for nondimensionalization of these equation are explained, which is not widely presented in the literature. The physical meaning of all nondimensional numbers arising as a result of nondimensionalization is discussed. 

The main goal of this survey is to demonstrate the underlying physical reasoning behind nondimensionalization and to obtain a nondimensional form of the equations, commonly used in numerical modeling, that depends only on a single parameter: the Rayleigh number.

\end{abstract}

\maketitle

%    Dedication.  If the dedication is longer than a line or two,
%    remove the centering instructions and the line break.
%\cleardoublepage
%\thispagestyle{empty}
%\vspace*{13.5pc}
%\begin{center}
%  Dedication text (use \\[2pt] for line break if necessary)
%\end{center}
%\cleardoublepage

%    Change page number to 6 if a dedication is present.
\setcounter{page}{4}
\setcounter{tocdepth}{2}

\tableofcontents

%    Include unnumbered chapters (preface, acknowledgments, etc.) here.
% 

\providecommand{\norm}[1]{\lVert#1 \rVert}
\providecommand{\R}{\begin{pmatrix} R \\ 0 \end{pmatrix}}
\providecommand{\Q}{\begin{pmatrix} Q_1^{T} \\ Q_2^{T} \end{pmatrix}}
\providecommand{\SVD}{\begin{pmatrix} \Sigma \\ 0 \end{pmatrix}}
\providecommand{\SVDr}{\begin{pmatrix} \Sigma_1 & 0 \\ 0 & 0 \end{pmatrix}}
\providecommand{\V}{\begin{pmatrix} V_1^{T} \\ V_2^{T} \end{pmatrix}}
\providecommand{\U}{\begin{pmatrix} U_1  U_2 \end{pmatrix}}
\providecommand{\Vr}{\begin{pmatrix} v_1^{T} \\ \vdots \\ v_r^{T} \end{pmatrix}}
\providecommand{\Vn}{\begin{pmatrix} v_1^{T} \\ \vdots \\ v_n^{T} \end{pmatrix}}
\providecommand{\Vni}{\begin{pmatrix} v_{1i} \dots v_{ni}\end{pmatrix}}
\providecommand{\B}{\begin{pmatrix} b_1^{T} \\ \vdots \\ b_n^{T} \end{pmatrix}}

% Dimensional

\providecommand{\divergence}{\nabla\cdot}

\providecommand{\velocity}{\mathbf{v}}
\providecommand{\substDerivVel}{\frac{D\velocity}{dt}}
\providecommand{\partialTimeVel}{\partial_t\velocity}
\providecommand{\inertTermVel}{(\velocity \cdot \nabla)\velocity}

\providecommand{\velocityPressureTensor}{\partial_kv^l\partial_kv^l}

\providecommand{\surfaceNormal}{\mathbf{e}_z}
\providecommand{\pressGrad}{\nabla p}
\providecommand{\pressLaplacian}{\Delta p}
\providecommand{\laplacianVel}{\Delta\velocity}

% Nondimensional

\providecommand{\divergenceNondim}{\nabla^{\prime}\cdot}

\providecommand{\velocityNondim}{\mathbf{v^{\prime}}}

\providecommand{\substDerivVelNondim}{\frac{D\velocity^{\prime}}{dt^{\prime}}}
\providecommand{\partialTimeVelNondim}{\partial_{t^{\prime}}\velocity^{\prime}}
\providecommand{\inertTermVelNondim}{(\velocity^{\prime} \cdot \nabla^{\prime})\velocity^{\prime}}

\providecommand{\velocityPressureTensorNondim}{\partial_k^{\prime}v^{l \prime}\partial_k^{\prime}v^{l \prime}}

\providecommand{\pressGradNondim}{\nabla^{\prime} p^{\prime}}
\providecommand{\pressLaplacianNondim}{\Delta^{\prime} p^{\prime}}
\providecommand{\laplacianVelNondim}{\Delta^{\prime}\velocity^{\prime}}


\section{Introduction}

The purpose of this chapter is to derive the nondimensional form of the incompressible Navier-Stokes equations. Special attention is devoted to the physical meaning of the nondimensional parameters. It is shown that the procedure of nondimensionalization and interpretation of its results depend crucially on the choice of perspective, which can be either that of inertial flow, or of viscous flow. Both perspectives are discussed and their importance explained. 

\section{Navier-Stokes Equations}

The flow of an incompressible viscous fluid in a homogeneous gravitational field is described by the following set of equations known as Navier-Stokes equations for incompressible flow\cite{gratz}:
\begin{equation} \label{NS1}
\partialTimeVel + \inertTermVel = -g\surfaceNormal - \frac{1}{\rho}\pressGrad + \nu\laplacianVel 
\end{equation}
\begin{equation} \label{NS2}
\divergence\velocity = 0  
\end{equation}

Where

\begin{itemize}

\item[] $\surfaceNormal$ is the unit normal to the surface of the Earth.
\item[] $ [\rho] = M L^{-3} $ is density assumed constant.
\item[] $[g] = LT^{-2} $ is the acceleration due to uniform gravitational field.
\item[] $ [\velocity] = L T^{-1} $ is the velocity field of the flow.
\item[] $ [\nu] =[\frac{\mu}{\rho}] = L^{2} T^{-1} $ is the kinematic viscosity of the fluid assumed constant.
\item[] $ [p] = M L^{-1} T^{-2} $ is \emph{dynamic} pressure (i.e., the one due to inertial movement of fluid, which should not be confused with thermodynamic pressure).

\end{itemize}

Naturally, these equations must be supplemented by appropriate boundary conditions in order to describe a closed well-posed system. 

\section{Nondimensionalization of Navier-Stokes Equations}

Dimensionless quantities allow us to identify a family of similar physical processes, i.e. those which are invariant under the change of scale. Thus, \emph{dimensionless quantities parametrize the family of similar processes}. Since physical processes are approximately described by equations, it is natural to expect that the equations governing similar processes are also parametrized by dimensionless parameters. Therefore, it is possible to \emph{nondimensionalize} the equations and thus describe how processes change quantitatively in response to the change of dimensionless parameters. 

Naturally, it is advantageous to nondimensionalize the equations before they are solved numerically since all the information about the process's characteristics can be stored in a small number of dimensionless parameters, also reducing the possible confusion about which units to store the dimensional parameters in \cite{barenblatt1}. (cf. also \nameref{appendix_pi})

In order to non-dimensionalize the Navier-Stokes equations, it is necessary to introduce the characteristic parameters of the flow, namely:

\begin{itemize}

\item[] $ L $ - characteristic length scale.
\item[] $ V $ - characteristic velocity scale (e.g. average or maximum speed of the flow).
\item[] $ T $ - characteristic time scale.
\item[] $ P $ - characteristic pressure scale (e.g. pressure difference between the ends of a pipe).

\end{itemize}

Nondimensionalization is accomplished by a change of variables in which the original variables $t, \mathbf{x}, \velocity, p$ are expressed as fractions of the characteristic parameters of the flow:

$$ t^{\prime}  = \frac{t}{T},~ \velocityNondim = \frac{\velocity}{V},~ \mathbf{x^{\prime}} = \frac{\mathbf{x}}{L},~ p^{\prime} = \frac{p}{P} $$ 

Solving for the old dimensional variables and substituting them into (\ref{NS1}) and (\ref{NS2}) yields, dropping the primes for notational convenience:

\begin{equation} \label{NS1-preNondim}
\frac{V}{T} \partialTimeVel + \frac{V^2}{L} \inertTermVel = -g\surfaceNormal - \frac{P}{\rho L}\pressGrad + \frac{\nu V}{L^2}\laplacianVel 
\end{equation}
\begin{equation} \label{NS2-preNondim}
\divergence\velocity = 0 
\end{equation}

\subsection{Perspective of Nondimensionalization} \label{perspective}

To continue with nondimensionalization of the equations (\ref{NS1}) and (\ref{NS2}), it is necessary to choose a perspective. In other words, which factor are we most interested in considering our particular flow? 

In a case when we are interested in weighing the influence of inertia compared to other factors, the coefficients in the equation (\ref{NS1-preNondim}) must be rearranged so as to yield $1$ in front of the inertial term, namely $\inertTermVel$. Such flows arise in contexts when fluids are not viscous and the characteristic speed of the flow is very large, e.g. in a flow past an airfoil.

In other cases, it is interesting to weigh other factors influencing the flow against the effects of viscosity, such as in the problems of mantle convection. In that case, the coefficients must be rearranged so as to yield $1$ in front of the viscous term, namely $\laplacianVel$. 

\subsection{Inertial Flow Perspective}

According to \nameref{appendix_pi} \cite{barenblatt1}, the system can be described by $7 - 3 = 4$ nondimensional parameters, since there are $7$ dimensional parameters describing the flow ($\mathbf{x}, \velocity, \rho, \nu, g, p, t$) among which there are $3$ independent units of measurements: mass, length, and time ($M, L, T$).

From the point of view of the inertial terms, nondimensionalization leads to the following equations:

\begin{equation} \label{NS1-inertNondim}
St~ \partialTimeVel + \inertTermVel = -\frac{1}{2Fr} \surfaceNormal - \frac{1}{2 Eu}\pressGrad + \frac{1}{Re}\laplacianVel 
\end{equation}
\begin{equation} \label{NS2-inertNondim}
\divergence\velocity = 0 
\end{equation}

The four nondimensional parameters have a concrete physical meaning.

\subsubsection{Strouhal Number}

$$St = \frac{L}{VT}$$

is a measure of how stationary the flow is. It is a ratio of characteristic time of inertial motion and the characteristic time of the flow:

$$ St = \frac{\frac{L}{V}}{T} = \frac{T_{inertia}}{T}$$

For example, if the inertial flow of plasma is much faster than the period of oscillation of an external magnetic field ($ T_{inertia} \ll T$), then the fluid simply cannot measurably respond to such low-energy influence of the external field, and the flow can be treated as approximately stationary.

\subsubsection{Froude Number}

$$ Fr = \frac{V^2}{2gL} $$

is a measure of the influence of the gravitational field on the inertial flow. It can be interpreted as a ratio of kinetic energy density (i.e., energy density of the inertial flow) and the density of the work done by the gravitational field:

$$ Fr = \frac{V^2}{2gL}  = \frac{\rho V^2}{2\rho g L} = \frac{KE}{Work_{gravity}}$$

For example, large Froude numbers characterize the flow of water from a fire hose, where the jet is barely deflected from its rectilinear trajectory because of its high initial kinetic energy.

\subsubsection{Euler Number}

$$ Eu = \frac{\rho V^2}{2P} $$

is a measure of the influence of pressure on the inertial flow. It can be interpreted as a ratio of kinetic energy density and the work done by the pressure:

$$ Eu = \frac{\rho V^2}{2P} = \frac{KE}{Work_{pressure}} $$

The Euler number is large in highly inertial flows where pressure gradients do not significantly deflect the flow, for example, in waterfalls.

\subsubsection{Reynolds Number}

$$ Re = \frac{LV}{\nu} $$

is a measure of the influence of viscosity on the inertial flow. It can be interpreted as a ratio of time scales of viscous and inertial processes:

$$ Re = \frac{LV}{\nu}  = \frac{\frac{L^2}{\nu}}{\frac{L}{V}} = \frac{T_{viscosity}}{T_{inertia}}$$

Another way to interpret the Reynolds number is as the ratio of the characteristic kinetic energy to the characteristic viscous shear stress $\tau$ of the flow \cite[p. 484]{sivukhin}:

$$ KE \sim \rho V^2,~ \tau \sim \frac{\rho \nu V}{L},~ \frac{KE}{\tau} = \frac{LV}{\nu} = Re $$

In the case in which the energy dissipation due to viscosity is too slow ($ T_{inertia} \ll T_{viscosity}$), viscosity has little influence on the inertial motion of the fluid.

\subsubsection{Large Reynolds Numbers and Turbulence}

Thus, large Reynolds numbers correspond to small energy dissipation by internal friction in the fluid and indicate the dominance of the inertial flow. 

\emph{Turbulence} occurs precisely in such flows. Indeed, highly energetic physical systems are, in general, unstable. The dissipation of energy provides the mechanism for stabilization. Since for large Reynolds numbers viscosity does not dissipate adequate amounts of energy, a fundamentally different mechanism of energy dissipation is needed. Such mechanism is precisely provided by the developed turbulence, whereby the initial kinetic energy of the laminar flow is distributed on a spectrum of scales of stochastic pulsations of the fluid.

Thus, turbulence is caused by the interaction of smaller and larger scales of motion in the fluid, and these interactions provide a mechanism for energy dissipation. A remarkable example of the importance of turbulence in nature has been suggested by A.N. Kolmogorov\cite[pp. 185-186]{barenblatt_fracture}. He estimated that, in the absence of turbulence, the speed of the flow in the Volga river would be on the order of $400, 000$ miles per hour, which would make life completely impossible along its shores.

\subsection{Viscous Flow Perspective}

Equations (\ref{NS1-inertNondim}) and (\ref{NS2-inertNondim}) are suitable for eliminating negligible factors when the inertial flow is known to be dominant. For example, if the Reynolds number is large, then it is plausible to neglect the viscosity term in the equation. 

However, what if we are concerned with eliminating negligible factors knowing that the flow is inherently viscous? This context arises, for example, in problems involving mantle convection. Naturally, previous equations do not provide information about the relative importance of other factors compared to viscosity. Therefore, we must transform the equations (\ref{NS1-inertNondim}) and (\ref{NS2-inertNondim}) to reflect a viscous flow perspective.

From the perspective of viscous flow, the equations (\ref{NS1-inertNondim}) and (\ref{NS2-inertNondim}) become

\begin{equation} \label{NS1-viscNondim}
Re St~ \partialTimeVel + Re ~\inertTermVel = -\frac{Re}{2Fr} \surfaceNormal - \frac{Re}{2 Eu}\pressGrad + \laplacianVel 
\end{equation}
\begin{equation} \label{NS2-viscNondim}
\divergence\velocity = 0 
\end{equation}

The new dimensionless coefficients also have a physical interpretation.

\subsubsection{Reynolds Number Times Strouhal Number}

$$ ReSt = \frac{LV}{\nu} \frac{L}{VT} = \frac{L^2}{\nu T} $$

is a measure of how stationary the flow is, albeit now from a point of view of a viscous flow.

It can be interpreted as the ratio of the viscosity time scale to the characteristic time scale of the flow:

$$ ReSt = \frac{L^2}{\nu T} = \frac{\frac{L^2}{\nu}}{T} = \frac{T_{viscosity}}{T}$$

\subsubsection{Reynolds Number Over Froude Number}

$$ \frac{Re}{2Fr} = \frac{LV}{\nu}\frac{gL}{V^2} = \frac{gL^2}{\nu V} $$ 

is a measure of the influence of the gravitational field in a viscous flow. It can be interpreted as a ratio of viscosity and gravity time scales:

$$ \frac{Re}{2Fr} = \frac{gL^2}{\nu V} = \frac{\frac{L^2}{\nu}}{\frac{V}{g}} = \frac{T_{viscosity}}{T_{gravity}} $$ 

\subsubsection{Reynolds Number Over Euler Number}

$$ \frac{Re}{Eu} = \frac{LV}{\nu} \frac{P}{\rho V^2} = \frac{PL}{\rho\nu V}  = \frac{P}{\tau} $$

This coefficient can be interpreted as a ratio characterizing the balance of pressure and viscous shear stresses $\tau$ in the flow.

\subsubsection{Nondimensional Numbers as Ratios of Time Scales}

Because viscous dissipation is a process that takes place on a molecular scale, it is natural to expect the nondimensional equations to depend on ratios of time scales associated with microscopic processes. In contrast, in the case of the inertial perspective the nondimensional parameters were mostly ratios characterizing the energy balance of various macroscopic processes in the flow. 

This can be explained by the fact that microscopic processes are determined by local relaxation times. On the other hand, quantities like kinetic energy density are macroscopic and do not influence local relaxation times, which explains why they are not present in the nondimensional parameters when the viscous flow perspective is considered.

\subsection{Further Investigation of the System}

Thus, it is evident that nondimensionalization is not a deterministic process since its outcome depends on the additional information available to the researcher. 

Firstly, one must choose a perspective, or a point of view, in order to obtain useful information from the nondimensional forms of the equations. Otherwise, a common mistake may happen when terms that are not actually negligible are neglected because of the incorrect interpretation of the nondimensional numbers. For example, the limit of vanishing Reynolds number in the inertial perspective does not imply that the only remaining term in the equation is the viscous term. Indeed, from the perspective of viscosity, Reynolds number is multiplied by other dimensionless numbers in front of other terms such as the gradient of pressure, so the asymptotics of these compound terms must now be considered as well. 

Secondly, without further specification of the system it is impossible to simplify the equations further. Either one must estimate the nondimensional numbers by direct measurements of characteristic parameters of the flow, or introduce additional relationships between them. For example, the viscosity time scale can be chosen as a characteristic time scale, and characteristic pressure can be identified with the characteristic viscous shear stress of the flow.

\mainmatter
%    Include main chapters here.

\chapter*{Introduction}

Nondimensionalization is a powerful method of investigating complex systems when the exact dynamics is practically impossible to compute yet it is still possible to estimate the mutual importance of various factors on the overall dynamics. For example, although, in general, the exact solution to the Navier-Stokes equations cannot be computed, nondimensionalization of these equations may reveal the relative importance of pressure, viscosity, and other factors on the overall behavior of the flow. Thus, knowing that the dissipation of energy and momentum due to viscosity is negligible compared to the vigor of the inertial motion of the fluid, which manifests itself in a very large value of a nondimensional parameter called Reynolds number, it is possible to neglect the viscous terms in the Navier-Stokes equations, thus turning them into Euler's equations. Analagously, when Reynolds number is small and inertial motion is negligible compared to viscous effects, Navier-Stokes equations turn into Stokes equations. These examples demonstrate how nondimensionalization may serve as a tool leading to useful approximations and deeper understanding of the physics underlying mathematical equations.

At a first glance, nondimensionalization appears to be a simple scheme: first, a change of variables that would make the physical variables involved in the equations nondimensional, followed by simple algebraic manipulations with the coefficients of the equations and, finally, taking limits with respect to certain coefficients of interest, such as Reynolds number mentioned above. However, such unintelligent approach may lead to mistakes. For example, under incorrect choice of scales, it may lead to disappearance of the pressure term in the Stokes equations when they are derived from Navier-Stokes, which totally contradicts the physical fact that pressure is an important factor even for highly viscous flows. Thus, correct nondimensionalization requires understanding of the physics underlying the decision one makes when a certain scale must be chosen in order to nondimensionalize a particular physical quantity. Otherwise, nondimensionalization becomes a dull set of tricks, which cannot amount to genuine knowledge and understanding of the natural world. 

Unfortunately, although nondimensionalization has been successfully employed for a long time as a powerful method of investigation, no universal algorithm was developed for applying this method to all of the possible physical situations. It is natural, since it is impossible to account for all of the possible combinations of physical factors in an abstract mathematical scheme. Thus, when one wishes to nondimensionalize the equations of mantle convection, the first difficulty that arises is an absence of a reliable source devoted to this specific context. Because of that, a researcher is compelled to search through a vast, inhomogenous array of literature in order to answer questions regarding the nondimensionalization of the equations of mantle convection. This is obvously inefficient and the valuable knowledge is lost if a successful researcher never shares his synthesis with the community.

The present work is attempts to  address this very problem. It is a synthesis of the numerous, oftentimes very little but important ideas from the sources the author encountered. It is hoped to be a point of growth in the mantle convection community, providing a source that explains nondimensionalization in the context of mantle convection. The author hopes that this work save precious time for other researches and students, which should be devoted to solving new interesting problems. 

The goal of this work is to derive a nondimensional form of the equations of the Boussinesq approximation in the context of mantle convection in a form that depends on a single nondimensional parameter: the Rayleigh number. To do so, we start nondimensionalize the incompressible Stokes equations and the advection-diffusion equation for temperature, explaining the physical meaning of this procedure and its underlying subtleties. Only the simplest form of the equations is considered in which the viscosity and thermal diffusivity are constant. This simplest case is, nevertheless, important, since it is often used for validating competing computational software that solve the same problem employing various numerical methodologies and algorithms.

In addition to presenting nondimensionalization of the equations commonly used in the mantle convection community, the author hopes that his work is also useful from a pedagogical perspective. Hopefully, it answers the questions that members of the community, especially students, often have, and, even more importantly, that it explains how the answers are actually found.

\thanks{The author would like to thank Professor Donald L. Turcotte, and Professor Louise Kellogg from Earth and Planetary Sciences, and Professor Elbridge G. Puckett from the Department of Mathematics and the Computational Infrastructure for Geodynamics at UC Davis for their comments, suggestions, and support for this work.}

% \chapter{Incompressible Navier-Stokes Equations} \label{navier_stokes}
% 

\providecommand{\norm}[1]{\lVert#1 \rVert}
\providecommand{\R}{\begin{pmatrix} R \\ 0 \end{pmatrix}}
\providecommand{\Q}{\begin{pmatrix} Q_1^{T} \\ Q_2^{T} \end{pmatrix}}
\providecommand{\SVD}{\begin{pmatrix} \Sigma \\ 0 \end{pmatrix}}
\providecommand{\SVDr}{\begin{pmatrix} \Sigma_1 & 0 \\ 0 & 0 \end{pmatrix}}
\providecommand{\V}{\begin{pmatrix} V_1^{T} \\ V_2^{T} \end{pmatrix}}
\providecommand{\U}{\begin{pmatrix} U_1  U_2 \end{pmatrix}}
\providecommand{\Vr}{\begin{pmatrix} v_1^{T} \\ \vdots \\ v_r^{T} \end{pmatrix}}
\providecommand{\Vn}{\begin{pmatrix} v_1^{T} \\ \vdots \\ v_n^{T} \end{pmatrix}}
\providecommand{\Vni}{\begin{pmatrix} v_{1i} \dots v_{ni}\end{pmatrix}}
\providecommand{\B}{\begin{pmatrix} b_1^{T} \\ \vdots \\ b_n^{T} \end{pmatrix}}

% Dimensional

\providecommand{\divergence}{\nabla\cdot}

\providecommand{\velocity}{\mathbf{v}}
\providecommand{\substDerivVel}{\frac{D\velocity}{dt}}
\providecommand{\partialTimeVel}{\partial_t\velocity}
\providecommand{\inertTermVel}{(\velocity \cdot \nabla)\velocity}

\providecommand{\velocityPressureTensor}{\partial_kv^l\partial_kv^l}

\providecommand{\surfaceNormal}{\mathbf{e}_z}
\providecommand{\pressGrad}{\nabla p}
\providecommand{\pressLaplacian}{\Delta p}
\providecommand{\laplacianVel}{\Delta\velocity}

% Nondimensional

\providecommand{\divergenceNondim}{\nabla^{\prime}\cdot}

\providecommand{\velocityNondim}{\mathbf{v^{\prime}}}

\providecommand{\substDerivVelNondim}{\frac{D\velocity^{\prime}}{dt^{\prime}}}
\providecommand{\partialTimeVelNondim}{\partial_{t^{\prime}}\velocity^{\prime}}
\providecommand{\inertTermVelNondim}{(\velocity^{\prime} \cdot \nabla^{\prime})\velocity^{\prime}}

\providecommand{\velocityPressureTensorNondim}{\partial_k^{\prime}v^{l \prime}\partial_k^{\prime}v^{l \prime}}

\providecommand{\pressGradNondim}{\nabla^{\prime} p^{\prime}}
\providecommand{\pressLaplacianNondim}{\Delta^{\prime} p^{\prime}}
\providecommand{\laplacianVelNondim}{\Delta^{\prime}\velocity^{\prime}}


\section{Introduction}

The purpose of this chapter is to derive the nondimensional form of the incompressible Navier-Stokes equations. Special attention is devoted to the physical meaning of the nondimensional parameters. It is shown that the procedure of nondimensionalization and interpretation of its results depend crucially on the choice of perspective, which can be either that of inertial flow, or of viscous flow. Both perspectives are discussed and their importance explained. 

\section{Navier-Stokes Equations}

The flow of an incompressible viscous fluid in a homogeneous gravitational field is described by the following set of equations known as Navier-Stokes equations for incompressible flow\cite{gratz}:
\begin{equation} \label{NS1}
\partialTimeVel + \inertTermVel = -g\surfaceNormal - \frac{1}{\rho}\pressGrad + \nu\laplacianVel 
\end{equation}
\begin{equation} \label{NS2}
\divergence\velocity = 0  
\end{equation}

Where

\begin{itemize}

\item[] $\surfaceNormal$ is the unit normal to the surface of the Earth.
\item[] $ [\rho] = M L^{-3} $ is density assumed constant.
\item[] $[g] = LT^{-2} $ is the acceleration due to uniform gravitational field.
\item[] $ [\velocity] = L T^{-1} $ is the velocity field of the flow.
\item[] $ [\nu] =[\frac{\mu}{\rho}] = L^{2} T^{-1} $ is the kinematic viscosity of the fluid assumed constant.
\item[] $ [p] = M L^{-1} T^{-2} $ is \emph{dynamic} pressure (i.e., the one due to inertial movement of fluid, which should not be confused with thermodynamic pressure).

\end{itemize}

Naturally, these equations must be supplemented by appropriate boundary conditions in order to describe a closed well-posed system. 

\section{Nondimensionalization of Navier-Stokes Equations}

Dimensionless quantities allow us to identify a family of similar physical processes, i.e. those which are invariant under the change of scale. Thus, \emph{dimensionless quantities parametrize the family of similar processes}. Since physical processes are approximately described by equations, it is natural to expect that the equations governing similar processes are also parametrized by dimensionless parameters. Therefore, it is possible to \emph{nondimensionalize} the equations and thus describe how processes change quantitatively in response to the change of dimensionless parameters. 

Naturally, it is advantageous to nondimensionalize the equations before they are solved numerically since all the information about the process's characteristics can be stored in a small number of dimensionless parameters, also reducing the possible confusion about which units to store the dimensional parameters in \cite{barenblatt1}. (cf. also \nameref{appendix_pi})

In order to non-dimensionalize the Navier-Stokes equations, it is necessary to introduce the characteristic parameters of the flow, namely:

\begin{itemize}

\item[] $ L $ - characteristic length scale.
\item[] $ V $ - characteristic velocity scale (e.g. average or maximum speed of the flow).
\item[] $ T $ - characteristic time scale.
\item[] $ P $ - characteristic pressure scale (e.g. pressure difference between the ends of a pipe).

\end{itemize}

Nondimensionalization is accomplished by a change of variables in which the original variables $t, \mathbf{x}, \velocity, p$ are expressed as fractions of the characteristic parameters of the flow:

$$ t^{\prime}  = \frac{t}{T},~ \velocityNondim = \frac{\velocity}{V},~ \mathbf{x^{\prime}} = \frac{\mathbf{x}}{L},~ p^{\prime} = \frac{p}{P} $$ 

Solving for the old dimensional variables and substituting them into (\ref{NS1}) and (\ref{NS2}) yields, dropping the primes for notational convenience:

\begin{equation} \label{NS1-preNondim}
\frac{V}{T} \partialTimeVel + \frac{V^2}{L} \inertTermVel = -g\surfaceNormal - \frac{P}{\rho L}\pressGrad + \frac{\nu V}{L^2}\laplacianVel 
\end{equation}
\begin{equation} \label{NS2-preNondim}
\divergence\velocity = 0 
\end{equation}

\subsection{Perspective of Nondimensionalization} \label{perspective}

To continue with nondimensionalization of the equations (\ref{NS1}) and (\ref{NS2}), it is necessary to choose a perspective. In other words, which factor are we most interested in considering our particular flow? 

In a case when we are interested in weighing the influence of inertia compared to other factors, the coefficients in the equation (\ref{NS1-preNondim}) must be rearranged so as to yield $1$ in front of the inertial term, namely $\inertTermVel$. Such flows arise in contexts when fluids are not viscous and the characteristic speed of the flow is very large, e.g. in a flow past an airfoil.

In other cases, it is interesting to weigh other factors influencing the flow against the effects of viscosity, such as in the problems of mantle convection. In that case, the coefficients must be rearranged so as to yield $1$ in front of the viscous term, namely $\laplacianVel$. 

\subsection{Inertial Flow Perspective}

According to \nameref{appendix_pi} \cite{barenblatt1}, the system can be described by $7 - 3 = 4$ nondimensional parameters, since there are $7$ dimensional parameters describing the flow ($\mathbf{x}, \velocity, \rho, \nu, g, p, t$) among which there are $3$ independent units of measurements: mass, length, and time ($M, L, T$).

From the point of view of the inertial terms, nondimensionalization leads to the following equations:

\begin{equation} \label{NS1-inertNondim}
St~ \partialTimeVel + \inertTermVel = -\frac{1}{2Fr} \surfaceNormal - \frac{1}{2 Eu}\pressGrad + \frac{1}{Re}\laplacianVel 
\end{equation}
\begin{equation} \label{NS2-inertNondim}
\divergence\velocity = 0 
\end{equation}

The four nondimensional parameters have a concrete physical meaning.

\subsubsection{Strouhal Number}

$$St = \frac{L}{VT}$$

is a measure of how stationary the flow is. It is a ratio of characteristic time of inertial motion and the characteristic time of the flow:

$$ St = \frac{\frac{L}{V}}{T} = \frac{T_{inertia}}{T}$$

For example, if the inertial flow of plasma is much faster than the period of oscillation of an external magnetic field ($ T_{inertia} \ll T$), then the fluid simply cannot measurably respond to such low-energy influence of the external field, and the flow can be treated as approximately stationary.

\subsubsection{Froude Number}

$$ Fr = \frac{V^2}{2gL} $$

is a measure of the influence of the gravitational field on the inertial flow. It can be interpreted as a ratio of kinetic energy density (i.e., energy density of the inertial flow) and the density of the work done by the gravitational field:

$$ Fr = \frac{V^2}{2gL}  = \frac{\rho V^2}{2\rho g L} = \frac{KE}{Work_{gravity}}$$

For example, large Froude numbers characterize the flow of water from a fire hose, where the jet is barely deflected from its rectilinear trajectory because of its high initial kinetic energy.

\subsubsection{Euler Number}

$$ Eu = \frac{\rho V^2}{2P} $$

is a measure of the influence of pressure on the inertial flow. It can be interpreted as a ratio of kinetic energy density and the work done by the pressure:

$$ Eu = \frac{\rho V^2}{2P} = \frac{KE}{Work_{pressure}} $$

The Euler number is large in highly inertial flows where pressure gradients do not significantly deflect the flow, for example, in waterfalls.

\subsubsection{Reynolds Number}

$$ Re = \frac{LV}{\nu} $$

is a measure of the influence of viscosity on the inertial flow. It can be interpreted as a ratio of time scales of viscous and inertial processes:

$$ Re = \frac{LV}{\nu}  = \frac{\frac{L^2}{\nu}}{\frac{L}{V}} = \frac{T_{viscosity}}{T_{inertia}}$$

Another way to interpret the Reynolds number is as the ratio of the characteristic kinetic energy to the characteristic viscous shear stress $\tau$ of the flow \cite[p. 484]{sivukhin}:

$$ KE \sim \rho V^2,~ \tau \sim \frac{\rho \nu V}{L},~ \frac{KE}{\tau} = \frac{LV}{\nu} = Re $$

In the case in which the energy dissipation due to viscosity is too slow ($ T_{inertia} \ll T_{viscosity}$), viscosity has little influence on the inertial motion of the fluid.

\subsubsection{Large Reynolds Numbers and Turbulence}

Thus, large Reynolds numbers correspond to small energy dissipation by internal friction in the fluid and indicate the dominance of the inertial flow. 

\emph{Turbulence} occurs precisely in such flows. Indeed, highly energetic physical systems are, in general, unstable. The dissipation of energy provides the mechanism for stabilization. Since for large Reynolds numbers viscosity does not dissipate adequate amounts of energy, a fundamentally different mechanism of energy dissipation is needed. Such mechanism is precisely provided by the developed turbulence, whereby the initial kinetic energy of the laminar flow is distributed on a spectrum of scales of stochastic pulsations of the fluid.

Thus, turbulence is caused by the interaction of smaller and larger scales of motion in the fluid, and these interactions provide a mechanism for energy dissipation. A remarkable example of the importance of turbulence in nature has been suggested by A.N. Kolmogorov\cite[pp. 185-186]{barenblatt_fracture}. He estimated that, in the absence of turbulence, the speed of the flow in the Volga river would be on the order of $400, 000$ miles per hour, which would make life completely impossible along its shores.

\subsection{Viscous Flow Perspective}

Equations (\ref{NS1-inertNondim}) and (\ref{NS2-inertNondim}) are suitable for eliminating negligible factors when the inertial flow is known to be dominant. For example, if the Reynolds number is large, then it is plausible to neglect the viscosity term in the equation. 

However, what if we are concerned with eliminating negligible factors knowing that the flow is inherently viscous? This context arises, for example, in problems involving mantle convection. Naturally, previous equations do not provide information about the relative importance of other factors compared to viscosity. Therefore, we must transform the equations (\ref{NS1-inertNondim}) and (\ref{NS2-inertNondim}) to reflect a viscous flow perspective.

From the perspective of viscous flow, the equations (\ref{NS1-inertNondim}) and (\ref{NS2-inertNondim}) become

\begin{equation} \label{NS1-viscNondim}
Re St~ \partialTimeVel + Re ~\inertTermVel = -\frac{Re}{2Fr} \surfaceNormal - \frac{Re}{2 Eu}\pressGrad + \laplacianVel 
\end{equation}
\begin{equation} \label{NS2-viscNondim}
\divergence\velocity = 0 
\end{equation}

The new dimensionless coefficients also have a physical interpretation.

\subsubsection{Reynolds Number Times Strouhal Number}

$$ ReSt = \frac{LV}{\nu} \frac{L}{VT} = \frac{L^2}{\nu T} $$

is a measure of how stationary the flow is, albeit now from a point of view of a viscous flow.

It can be interpreted as the ratio of the viscosity time scale to the characteristic time scale of the flow:

$$ ReSt = \frac{L^2}{\nu T} = \frac{\frac{L^2}{\nu}}{T} = \frac{T_{viscosity}}{T}$$

\subsubsection{Reynolds Number Over Froude Number}

$$ \frac{Re}{2Fr} = \frac{LV}{\nu}\frac{gL}{V^2} = \frac{gL^2}{\nu V} $$ 

is a measure of the influence of the gravitational field in a viscous flow. It can be interpreted as a ratio of viscosity and gravity time scales:

$$ \frac{Re}{2Fr} = \frac{gL^2}{\nu V} = \frac{\frac{L^2}{\nu}}{\frac{V}{g}} = \frac{T_{viscosity}}{T_{gravity}} $$ 

\subsubsection{Reynolds Number Over Euler Number}

$$ \frac{Re}{Eu} = \frac{LV}{\nu} \frac{P}{\rho V^2} = \frac{PL}{\rho\nu V}  = \frac{P}{\tau} $$

This coefficient can be interpreted as a ratio characterizing the balance of pressure and viscous shear stresses $\tau$ in the flow.

\subsubsection{Nondimensional Numbers as Ratios of Time Scales}

Because viscous dissipation is a process that takes place on a molecular scale, it is natural to expect the nondimensional equations to depend on ratios of time scales associated with microscopic processes. In contrast, in the case of the inertial perspective the nondimensional parameters were mostly ratios characterizing the energy balance of various macroscopic processes in the flow. 

This can be explained by the fact that microscopic processes are determined by local relaxation times. On the other hand, quantities like kinetic energy density are macroscopic and do not influence local relaxation times, which explains why they are not present in the nondimensional parameters when the viscous flow perspective is considered.

\subsection{Further Investigation of the System}

Thus, it is evident that nondimensionalization is not a deterministic process since its outcome depends on the additional information available to the researcher. 

Firstly, one must choose a perspective, or a point of view, in order to obtain useful information from the nondimensional forms of the equations. Otherwise, a common mistake may happen when terms that are not actually negligible are neglected because of the incorrect interpretation of the nondimensional numbers. For example, the limit of vanishing Reynolds number in the inertial perspective does not imply that the only remaining term in the equation is the viscous term. Indeed, from the perspective of viscosity, Reynolds number is multiplied by other dimensionless numbers in front of other terms such as the gradient of pressure, so the asymptotics of these compound terms must now be considered as well. 

Secondly, without further specification of the system it is impossible to simplify the equations further. Either one must estimate the nondimensional numbers by direct measurements of characteristic parameters of the flow, or introduce additional relationships between them. For example, the viscosity time scale can be chosen as a characteristic time scale, and characteristic pressure can be identified with the characteristic viscous shear stress of the flow.

% \chapter{Stokes Equations} \label{stokes}
% 
\providecommand{\norm}[1]{\lVert#1 \rVert}
\providecommand{\R}{\begin{pmatrix} R \\ 0 \end{pmatrix}}
\providecommand{\Q}{\begin{pmatrix} Q_1^{T} \\ Q_2^{T} \end{pmatrix}}
\providecommand{\SVD}{\begin{pmatrix} \Sigma \\ 0 \end{pmatrix}}
\providecommand{\SVDr}{\begin{pmatrix} \Sigma_1 & 0 \\ 0 & 0 \end{pmatrix}}
\providecommand{\V}{\begin{pmatrix} V_1^{T} \\ V_2^{T} \end{pmatrix}}
\providecommand{\U}{\begin{pmatrix} U_1  U_2 \end{pmatrix}}
\providecommand{\Vr}{\begin{pmatrix} v_1^{T} \\ \vdots \\ v_r^{T} \end{pmatrix}}
\providecommand{\Vn}{\begin{pmatrix} v_1^{T} \\ \vdots \\ v_n^{T} \end{pmatrix}}
\providecommand{\Vni}{\begin{pmatrix} v_{1i} \dots v_{ni}\end{pmatrix}}
\providecommand{\B}{\begin{pmatrix} b_1^{T} \\ \vdots \\ b_n^{T} \end{pmatrix}}

% Dimensional

\providecommand{\divergence}{\nabla\cdot}

\providecommand{\velocity}{\mathbf{v}}
\providecommand{\substDerivVel}{\frac{D\velocity}{dt}}
\providecommand{\partialTimeVel}{\partial_t\velocity}
\providecommand{\inertTermVel}{(\velocity\nabla)\velocity}

\providecommand{\velocityPressureTensor}{\partial_kv^l\partial_kv^l}

\providecommand{\surfaceNormal}{\mathbf{e}_z}
\providecommand{\pressGrad}{\nabla p}
\providecommand{\pressLaplacian}{\Delta p}
\providecommand{\laplacianVel}{\Delta\velocity}

% Nondimensional

\providecommand{\divergenceNondim}{\nabla^{\prime}\cdot}

\providecommand{\velocityNondim}{\mathbf{v^{\prime}}}

\providecommand{\substDerivVelNondim}{\frac{D\velocity^{\prime}}{dt^{\prime}}}
\providecommand{\partialTimeVelNondim}{\partial_{t^{\prime}}\velocity^{\prime}}
\providecommand{\inertTermVelNondim}{(\velocity^{\prime}\nabla^{\prime})\velocity^{\prime}}

\providecommand{\velocityPressureTensorNondim}{\partial_k^{\prime}v^{l \prime}\partial_k^{\prime}v^{l \prime}}

\providecommand{\pressGradNondim}{\nabla^{\prime} p^{\prime}}
\providecommand{\pressLaplacianNondim}{\Delta^{\prime} p^{\prime}}
\providecommand{\laplacianVelNondim}{\Delta^{\prime}\velocity^{\prime}}

\section{Introduction}
 
The incompressible Stokes equations describing creeping or highly viscous flows are derived from the incompressible Navier-Stokes equations. The derivation is done from the viscous flow perspective, since the inertial flow perspective is inappropriate for creeping flows. The physical meaning of the choice of pressure scale based on viscous shear stress is fully explained, which is commonly left unexplained in the literature. The physical meaning of the nondimensional parameters arising from the nondimensionalization is discussed.

\section{Low Reynolds Number Flows}

\subsection{Physical Meaning of Small Reynolds Number}

A very small Reynolds number is indicative of the dominance of viscosity in the flow. Therefore, it is natural to consider the Navier-Stokes equations from the perspective of viscosity:

\begin{equation} \label{NS1-viscNondim}
Re St~ \partialTimeVel + Re ~\inertTermVel = -\frac{Re}{2Fr} \surfaceNormal - \frac{Re}{2 Eu}\pressGrad + \laplacianVel 
\end{equation}
\begin{equation} \label{NS2-viscNondim}
\divergence\velocity = 0 
\end{equation}

What does it mean physically for the Reynolds number to be small? 
$$ Re = \frac{LV}{\nu}  = \frac{\frac{L^2}{\nu}}{\frac{L}{V}} = \frac{T_{viscosity}}{T_{inertia}} \ll 1 \Longleftrightarrow T_{viscosity} \ll T_{inertia} $$

This means that the inertial flow is much slower than the processes associated with viscosity.

Another way to interpret a small Reynolds number is when viscous shear stress $\tau$ dissipates most of the kinetic energy of the inertial flow:

$$ KE \sim \rho V^2,~ \tau \sim \frac{\rho \nu V}{L},~ \frac{KE}{\tau} = \frac{LV}{\nu} = Re \ll 1 \Longleftrightarrow KE \ll \tau$$

This means that the energy dissipation due to viscous shear stresses dominates the kinetic energy in the energy balance of the flow \cite[p. 484]{sivukhin}.

\subsubsection{Choice of Time Scale}

Small Reynolds number means that the viscous dissipation is a much faster process than the inertial flow of the fluid's particles:

$$ Re = \frac{T_{viscosity}}{T_{inertia}} \ll 1 \implies T_{viscosity} \ll T_{inertia} $$

In the limit $ Re \longrightarrow 0 \implies T_{visc} = 0$, which means that viscous dissipation happens infinitely fast. Naturally, infinitely fast processes cannot serve as a measure of processes that take finite time to evolve. Therefore, the appropriate characteristic time scale of the system is the time scale of the inertial flow:

$$ T = T_{inertia} = \frac{L}{V} $$

\subsubsection{Choice of Pressure Scale}

The choice of the characteristic scale for the pressure is based on the consideration that for low Reynolds number flows the kinetic energy is negligible compared to the viscous shear stress. Therefore, the estimate of a characteristic viscous shear stress is adopted as a characteristic pressure scale.


There are at least two ways to arrive at an estimate of the viscous shear stress.

The first is to consider Newton's approximation to the force $F$ due to viscosity\cite{zorich}. The surface force due of viscosity is directly proportional to the surface area (estimated as $L^2$) of contact between two layers of fluid, the speed of their relative motion $V$, and inversely proportional to the distance between the moving layer of the fluid and the layer that is at rest (estimated as $L$). The constant of proportionality is, by definition, the dynamic viscosity $\mu = \rho \nu$:

$$ F = \frac{\mu V L^2}{L} = \frac{\rho \nu V L^2}{L} = \rho \nu VL$$

Hence, the viscous shear stress, which is simply the force due to viscosity per area, is

$$ P = \frac{F}{L^2} = \frac{\rho \nu V L}{L^2} = \frac{\rho \nu V}{L}$$

Also, the viscous stress tensor $\tau_{ij}$ can be estimated based on its dimension: 

$$ \tau_{ij} = \rho \nu (\frac{\partial v_i}{\partial x_j} + \frac{\partial v_j}{\partial x_i}) \sim \frac{\rho \nu V}{L} = P $$

It is interesting to see why the choice of scale based on kinetic energy is inappropriate for low Reynolds number flows.

The equations (\ref{NS1-viscNondim}) and (\ref{NS2-viscNondim}) suggest two characteristic pressure scales:

$$ P_1 = \frac{2EuP}{Re} = \frac{\rho \nu V}{L}$$

and

$$ P_2 = 2EuP  = \rho V^2$$


Their ratio is equal to the Reynolds number:

$$ \frac{P_2}{P_1} = \rho V^2 \frac{L}{\rho \nu V} = \frac{LV}{\nu} = Re \Longleftrightarrow P_2 = Re P_1$$

Considering these choices of scale, the coefficient multiplying the gradient of pressure in the equation (\ref{NS1-viscNondim}) becomes either

$$ \frac{Re}{2 Eu} = \frac{P_1L}{\rho \nu V} = 1$$

or 

$$ \frac{Re}{2 Eu} = \frac{P_2L}{\rho \nu V} = Re \frac{P_1L}{\rho \nu V} = Re $$

If the scale $P_2$ is chosen, the pressure gradient term will vanish in the limit $ Re \longrightarrow 0$. This contradicts the physical fact that pressure influences low Reynolds number flows \cite[pp. 433-434]{leal}. 

Thus, the natural choice of characteristic pressure scale for low Reynolds number flows is the characteristic viscous shear stress:

$$ P = P_{viscosity} = \frac{\rho \nu V}{L} $$

\subsection{Estimation of Nondimensional Parameters} \label{stokes_parameters}

Given the choices of time and pressure scales discussed above, we can estimate the nondimensional coefficients in the equations (\ref{NS1-viscNondim}) and (\ref{NS2-viscNondim}).

\subsubsection{Strouhal Number}

Strouhal number 

$$ St = \frac{L}{VT} = 1 $$ 

The physical interpretation of Strouhal number being equal to unity is that the stationary and nonstationary components of the flow are in balance.

\subsubsection{Reynolds Number Over Froude Number}

$$ \frac{Re}{2Fr} = \frac{gL^2}{\nu V} = \frac{\rho gL}{\frac{\rho \nu V}{L}} = \frac{E_{gravity}}{P_{viscosity}} $$

This dimensionless number can be interpreted as a measure of the work done by the gravitational field on the fluid in comparison with the viscous shear stresses.

\subsubsection{Reynolds Number Over Euler Number}

$$ \frac{Re}{2Eu} = \frac{PL}{\rho \nu V} = 1 $$

This ratio can be interpreted as a balance of pressure and viscous shear stresses in the flow.

\subsubsection{Euler Number}

$$ Eu = \frac{\rho V^2}{2P} = \frac{\rho V^2}{2} \frac{L}{\rho\nu V} = \frac{Re}{2}$$

Euler number can be interpreted as a ratio characterizing energy balance between the kinetic energy of the flow and and viscous shear stresses.

\subsubsection{Nondimensional Equations}

Considering the estimations of the nondimensional parameters presented above, the equations (\ref{NS1-viscNondim}) and (\ref{NS2-viscNondim}) become:


\begin{equation} \label{NS1-viscNondim_scales}
Re(\partialTimeVel + \inertTermVel) = -\frac{gL^2}{\nu V} \surfaceNormal - \pressGrad + \laplacianVel 
\end{equation}
\begin{equation} \label{NS2-viscNondim_scales}
\divergence\velocity = 0 
\end{equation}

\subsection{Stokes Equations}

Now it is possible to derive the Stokes equations from the equations (\ref{NS1-viscNondim_scales}) and (\ref{NS2-viscNondim_scales}) by taking the limit of $ Re \longrightarrow 0$, which is a common approximation when creeping or very viscous flows are considered:

\begin{equation} \label{NS1-StokesNondim}
\laplacianVel = \frac{gL^2}{\nu V} \surfaceNormal + \pressGrad 
\end{equation}
\begin{equation} \label{NS2-StokesNondim}
\divergence\velocity = 0 
\end{equation}

\section{Passage to Dimensional Stokes Equations}

Since Stokes equations are commonly presented in a dimensional form, it is educational to demonstrate the underlying procedure that makes the dimensional form of the equations valid.

Firstly, equation (\ref{NS2-StokesNondim}) is brought back by multiplying it by 

$$ \frac{P}{L^2} = \frac{\rho \nu V}{L^3}$$

and the equation (\ref{NS1-StokesNondim}) becomes dimensional when multiplied by 

$$ \frac{\nu V}{L^2} = \frac{2g Fr}{Re}$$

which is all equivalent to a change of variables 

$$ t^{\prime}  = \frac{t}{T} = \frac{tV}{L},~ \velocityNondim = \frac{\velocity}{V},~ \mathbf{r^{\prime}} = \frac{\mathbf{x}}{L},~ p^{\prime} = \frac{p}{P} = \frac{pL}{\rho \nu V} $$  

By performing either of the above procedures, we obtain Stokes equations in a dimensional form:
\
\begin{equation} \label{NS1-Stokes}
\nu \laplacianVel = g \surfaceNormal + \frac{1}{\rho} \pressGrad 
\end{equation}
\begin{equation} \label{NS2-Stokes}
\divergence\velocity = 0
\end{equation}

\subsection{Stokes Equations vs. Navier-Stokes}

Thus, the only difference between Stokes and Navier-Stokes equations is the presence of the inertial terms. In addition, the Poisson equation for pressure becomes Laplace's equation [\ref{pressure_laplace}], i.e. a homogenous Poisson equation, which manifests the independence of pressure from velocity in highly viscous flows. Furthermore, Stokes equations are linear, which further simplifies their numerical solution.

These facts could be postulated from purely physical consideration, yet only dimensional analysis presented above truly justifies neglecting the inertial terms under the condition $$Re \ll 1 $$
% 
% \chapter{Temperature Advection-Diffusion Equation} \label{heat}
% 
\providecommand{\norm}[1]{\lVert#1 \rVert}
\providecommand{\R}{\begin{pmatrix} R \\ 0 \end{pmatrix}}
\providecommand{\Q}{\begin{pmatrix} Q_1^{T} \\ Q_2^{T} \end{pmatrix}}
\providecommand{\SVD}{\begin{pmatrix} \Sigma \\ 0 \end{pmatrix}}
\providecommand{\SVDr}{\begin{pmatrix} \Sigma_1 & 0 \\ 0 & 0 \end{pmatrix}}
\providecommand{\V}{\begin{pmatrix} V_1^{T} \\ V_2^{T} \end{pmatrix}}
\providecommand{\U}{\begin{pmatrix} U_1  U_2 \end{pmatrix}}
\providecommand{\Vr}{\begin{pmatrix} v_1^{T} \\ \vdots \\ v_r^{T} \end{pmatrix}}
\providecommand{\Vn}{\begin{pmatrix} v_1^{T} \\ \vdots \\ v_n^{T} \end{pmatrix}}
\providecommand{\Vni}{\begin{pmatrix} v_{1i} \dots v_{ni}\end{pmatrix}}
\providecommand{\B}{\begin{pmatrix} b_1^{T} \\ \vdots \\ b_n^{T} \end{pmatrix}}

% Dimensional

\providecommand{\divergence}{\nabla\cdot}

\providecommand{\velocity}{\mathbf{v}}
\providecommand{\substDerivVel}{\frac{D\velocity}{dt}}
\providecommand{\partialTimeVel}{\partial_t\velocity}
\providecommand{\inertTermVel}{(\velocity\cdot\nabla)\velocity}

\providecommand{\velocityPressureTensor}{\partial_kv^l\partial_kv^l}

\providecommand{\surfaceNormal}{\mathbf{e}_z}
\providecommand{\pressGrad}{\nabla p}
\providecommand{\pressLaplacian}{\Delta p}
\providecommand{\laplacianVel}{\Delta\velocity}


\providecommand{\partialTimeTemp}{\partial_t\theta}
\providecommand{\inertTermTemp}{(\velocity \cdot \nabla)\theta}
\providecommand{\tempGrad}{\nabla \theta}
\providecommand{\laplacianTemp}{\Delta\theta}

% Nondimensional

\providecommand{\divergenceNondim}{\nabla^{\prime}\cdot}

\providecommand{\velocityNondim}{\mathbf{v^{\prime}}}

\providecommand{\substDerivVelNondim}{\frac{D\velocity^{\prime}}{dt^{\prime}}}
\providecommand{\partialTimeVelNondim}{\partial_{t^{\prime}}\velocity^{\prime}}
\providecommand{\inertTermVelNondim}{(\velocity^{\prime} \cdot \nabla^{\prime})\velocity^{\prime}}

\providecommand{\velocityPressureTensorNondim}{\partial_k^{\prime}v^{l \prime}\partial_k^{\prime}v^{l \prime}}

\providecommand{\pressGradNondim}{\nabla^{\prime} p^{\prime}}
\providecommand{\pressLaplacianNondim}{\Delta^{\prime} p^{\prime}}
\providecommand{\laplacianVelNondim}{\Delta^{\prime}\velocity^{\prime}}


\section{Introduction}

The derivation of the temperature advection-diffusion based on the principle of conservation of energy is presented. Then, the temperature advection-diffusion is nondimensionalized from both the advection and conduction perspectives, and the physical meaning of the corresponding nondimensional parameters is discussed. 

\section{Derivation of Temperature Advection-Diffusion Equation}

The so-called temperature advection-diffusion equation, which is mathematically equivalent to the heat equation with the additional terms responsible for convective transport of heat, provides an approximate description of heat transfer processes in a thermally isotropic and uniform medium.\footnote{An alternative name is \emph{convection-diffusion} equation. In terms of semantics, advection is preferrable to convection in this context. Indeed, \emph{convection} means the flow of the fluid in response to the heat gradients present in the fluid. \emph{Advection}, however, is a term used to describe a transport of a quantity (such as temperature, concentration of a chemical, etc.) in response to the fluid motion. That is why, in this case, temperature is \emph{advected}. Of course, advection of temperature (heat) causes convection of a fluid an vice versa, so these proceses are interrelated, However, \emph{convective heat transfer} is often referred to as \emph{convection}, but, in the light of our previous remarks, it would be more appropriate to call it \emph{advection} and \emph{advective heat transport}. However, this is a terminological inconsistency rooted in history.} It describes the evolution of the temperature field in time. Remarkably, the same equation is used to describe mass transfer processes, since oftentimes the underlying physical assumptions are mathematically equivalent.

The assumptions underlying our derivation are the following.\cite{shubin}

\subsection{Constant Heat Capacity}

The amount of heat required for a substance of mass $m$ to change its temperature from $\theta_1$ to $\theta_2$ is directly proportional to $m$ and the increment of temperature $\theta_2 - \theta_1$:

$$ Q = cm(\theta_2 - \theta_1), ~[c] = \frac{L^2}{\Theta T^2} $$

The coefficient $c$ is called \emph{specific heat capacity} (i.e. heat capacity per unit mass).

\subsection{Fourier Law of Heat Conduction}

The amount of heat $\Delta Q$ transferred through an infinitesimal plate of area $\Delta S$ in time $\Delta t$ is directly proportional to $\Delta S$, $\Delta t$, and the rate of change of temperature along the unit normal to that plate $\tempGrad \cdot \mathbf{n}$:

$$ \Delta Q = -k\Delta S \Delta t ~ \tempGrad \cdot \mathbf{n}, ~ k > 0, ~[k] = \frac{ML}{T^3\Theta} $$

The coefficient $k$ is called \emph{thermal conductivity}. Note that the heat flows opposite to the increase of temperature, which explains the negative sign in the equation above.

\subsection{Conservation of Energy}

The rate of change of the amount of heat in a volume $\Omega \subset \mathbb{R}^3$ is 

$$ \frac{dQ}{dt} = \frac{d}{dt}\int \limits_\Omega \rho c~\theta dV = \rho c \int \limits_\Omega \frac{d\theta}{dt} dV$$

Note that $c$, $\rho$, and $k$ were assumed constant (uniform and isotropic medium), and the differentiation under the integral is allowed assuming that the temperature field $\theta$ is smooth enough. The last assumption is usually plausible since temperature transfer is a diffusive process which smoothens discontinuities.

Simultaneously, heat is flowing out of the volume $\Omega$ through its boundary $\partial \Omega$:

$$ \frac{dQ_{out}}{dt} = -\int \limits_{\partial \Omega} k (\tempGrad \cdot \mathbf{n}) ~dS$$

In other words, the amount of heat flowing out through the boundary per unit time is directly proportional to the negative flux of the gradient of temperature. 

By divergence theorem

$$ \int \limits_{\partial \Omega} (\tempGrad \cdot \mathbf{n}) ~dS = \int \limits_{\Omega} (\nabla \cdot \tempGrad) dV = \int \limits_{\Omega} \laplacianTemp dV \Longrightarrow  \frac{dQ_{out}}{dt} = - k\int \limits_{\Omega} \laplacianTemp dV $$

Conservation of energy requires that, given there are no sources of heat, the change of the amount of heat in the volume is equal to the amount of heat leaving or entering it through the bounday:

$$ \frac{dQ}{dt}  = -\frac{dQ_{out}}{dt} \Longleftrightarrow \rho c \int \limits_\Omega \frac{d\theta}{dt} dV = k\int \limits_{\Omega} \laplacianTemp dV $$

Since we assumed that the temperature field $\theta$ is smooth enough, and the volume $\Omega$ was arbitrary, the functions under the integral must be equal throughout the domain $\mathbb{R}^3$:

$$ \rho c\frac{d\theta}{dt} = k \laplacianTemp $$

Unfolding the material derivative $\frac{d\theta}{dt}$ and dividing the equation by $\rho c$ we obtain the advection-diffusion equation, describing  diffusion of temperature simultaneous with convective transport of heat by the velocity field. In other words, the equation takes both the heat conduction and advection processes into account:

\begin{equation} \label{heat_eq}
\partialTimeTemp + \inertTermTemp = \kappa \laplacianTemp 
\end{equation}


The coefficient $\kappa$ is called \emph{thermal diffusivity} and is a measure of the intensity of the diffusion of temperature:

$$ \kappa = \frac{k}{\rho c}, ~ [\kappa] = \frac{L^2}{T}$$

Naturally, boundary conditions must be supplied in order for the equation (\ref{heat_eq}) to describe a concrete system.

\section{Nondimensionalization of the Temperature Advection-Diffusion Equation}

The nondimensionalization of the temperature advection-diffusion (\ref{heat_eq}) is accomplished in the same manner as in the case of Navier-Stokes equations in chapter \ref{navier_stokes}. 

However, now it is necessary to introduce a new parameter, the characteristic temperature of the system $\Theta$. Operationally, it can be defined, for example, as the average temperature of the field or the difference between the maximum and the minimum temperatures.

According to Buckingham $\Pi$-theorem \cite{barenblatt1} (cf. also \nameref{appendix_pi}), the system can be described by $7 - 4 = 3$ nondimensional parameters, since there are $7$ dimensional parameters describing the heat transfer ($\mathbf{x}, \velocity, \rho, \nu, g, \theta, t$) among which there are $4$ independent units of measurements (mass, length, time, temperature).

By performing a change of variables 

$$ t^{\prime}  = \frac{t}{T},~ \velocityNondim = \frac{\velocity}{V},~ \mathbf{r^{\prime}} = \frac{\mathbf{x}}{L},~ \theta^{\prime} = \frac{\theta}{\Theta} $$ 

and substituting into the equation (\ref{heat_eq}), dropping the primes for notational convenience, we obtain

\begin{equation} \label{heat_eq_preNondim}
 \frac{\Theta}{T} \partialTimeTemp + \frac{\Theta V}{L} \inertTermTemp = \frac{\kappa \Theta}{L^2} \laplacianTemp 
\end{equation}

As in the case of Navier-Stokes equations in chapter \ref{navier_stokes}, it is necessary to choose a perspective in order to proceed with nondimensionalization.

\subsection{Advective Term Perspective}

From the point of view of the advective (inertial) term, the equation (\ref{heat_eq_preNondim}) becomes

\begin{equation} \label{heat_eq_inertNondim}
 St~ \partialTimeTemp + \inertTermTemp = \frac{1}{Pe} \laplacianTemp  
\end{equation}
 
Since Strouhal number $St$ has the same meaning as in nondimensionalization of Navier-Stokes equations discussed in chapter \ref{navier_stokes}, only the Peclet number $Pe$ deserves a discussion here.

\subsubsection{Peclet Number}

The Peclet number 

$$ Pe = \frac{LV}{\kappa} = \frac{\frac{L^2}{\kappa}}{\frac{L}{V}} = \frac{T_{conduction}}{T_{inertia}} $$

is a measure of the vigour of advection. It can be interpreted as a ratio of conduction and advection time scales, inertial time scale being equivalent to advection time scale. In other words, it indicates which mechanism of heat transfer, convective or conductive, is dominant in the flow.

It is possible to consider the Peclet number as a product of Reynolds and Prandtl number:

$$ Pe = \frac{LV}{\kappa} = \frac{LV}{\nu} \frac{\nu}{\kappa} = RePr, ~ Pr = \frac{\nu}{\kappa} $$

Since the Prandtl number characterizes the medium but not the flow itself, Peclet number can be viewed, to some degree, as a Reynolds number in the context of heat transfer.

\subsubsection{Prandtl Number}

Prandtl number

$$ Pr = \frac{\nu}{\kappa} = \frac{\frac{L^2}{\kappa}}{\frac{L^2}{\nu}} = \frac{T_{conduction}}{T_{viscosity}} $$

can be interpreted as a ratio of conduction and viscosity time scales. Since it does not involve any kinematic or dynamic quantities like characteristic length or speed, Prandtl number characterizes the properties of the medium but it does not characterize the properties of the flow. 

In other words, Prandtl number is constant for different types of flow in the same medium. It simply characterizes which mechanism of energy dissipation, viscous or heat conduction, is dominant for a given material.

\subsection{Conduction Term Perspective}

From the point of view of the conduction (diffusion) term, the equation (\ref{heat_eq_preNondim}) becomes

\begin{equation} \label{heat_eq_diffusNondim}
\frac{1}{Fo} \partialTimeTemp + Pe ~\inertTermTemp = \laplacianTemp  
\end{equation}

The only new dimensionless parameter is the Fourier number $Fo$.

\subsubsection{Fourier Number}

The Fourier number

$$ Fo = St Pe = \frac{\kappa T}{L^2} = \frac{T}{\frac{L^2}{\kappa}} = \frac{T}{T_{conduction}}$$

is a ratio of the characteristic time scale of the system and the conduction time scale. It is a measure of how stationary the heat transfer process is. The smaller the time scale of a certain process, the more dominant that process is in the information exchange in the system. 

For example, if the conduction time scale is much smaller than the characteristic time scale of the system (e.g. period of oscillation of a magnetic field), Fourier number will be large and the heat transfer can be considered stationary. In an opposite case, if the frequency, and hence the energy, of the alternating magnetic field is very large, than the fluid will respond to this disturbance and the heat transfer process will be nonstationary.

In other words, heat conduction, due to its dissipative nature, stabilizes the system, whereas external disturbances usually destabilize it, thus making the process inherently nonstationary.

% 
% \chapter{The Boussinesq Approximation} \label{boussinesq}
% \providecommand{\norm}[1]{\lVert#1 \rVert}
\providecommand{\R}{\begin{pmatrix} R \\ 0 \end{pmatrix}}
\providecommand{\Q}{\begin{pmatrix} Q_1^{T} \\ Q_2^{T} \end{pmatrix}}
\providecommand{\SVD}{\begin{pmatrix} \Sigma \\ 0 \end{pmatrix}}
\providecommand{\SVDr}{\begin{pmatrix} \Sigma_1 & 0 \\ 0 & 0 \end{pmatrix}}
\providecommand{\V}{\begin{pmatrix} V_1^{T} \\ V_2^{T} \end{pmatrix}}
\providecommand{\U}{\begin{pmatrix} U_1  U_2 \end{pmatrix}}
\providecommand{\Vr}{\begin{pmatrix} v_1^{T} \\ \vdots \\ v_r^{T} \end{pmatrix}}
\providecommand{\Vn}{\begin{pmatrix} v_1^{T} \\ \vdots \\ v_n^{T} \end{pmatrix}}
\providecommand{\Vni}{\begin{pmatrix} v_{1i} \dots v_{ni}\end{pmatrix}}
\providecommand{\B}{\begin{pmatrix} b_1^{T} \\ \vdots \\ b_n^{T} \end{pmatrix}}

% Dimensional

\providecommand{\divergence}{\nabla\cdot}

\providecommand{\velocity}{\mathbf{v}}
\providecommand{\substDerivVel}{\frac{D\velocity}{dt}}
\providecommand{\partialTimeVel}{\partial_t\velocity}
\providecommand{\inertTermVel}{(\velocity\nabla)\velocity}

\providecommand{\velocityPressureTensor}{\partial_kv^l\partial_kv^l}

\providecommand{\surfaceNormal}{\mathbf{e}_z}
\providecommand{\pressGrad}{\nabla p}
\providecommand{\pressLaplacian}{\Delta p}
\providecommand{\laplacianVel}{\Delta\velocity}


\providecommand{\partialTimeTemp}{\partial_t\theta}
\providecommand{\inertTermTemp}{(\velocity\nabla)\theta}
\providecommand{\tempGrad}{\nabla \theta}
\providecommand{\laplacianTemp}{\Delta\theta}

% Nondimensional

\providecommand{\divergenceNondim}{\nabla^{\prime}\cdot}

\providecommand{\velocityNondim}{\mathbf{v^{\prime}}}

\providecommand{\substDerivVelNondim}{\frac{D\velocity^{\prime}}{dt^{\prime}}}
\providecommand{\partialTimeVelNondim}{\partial_{t^{\prime}}\velocity^{\prime}}
\providecommand{\inertTermVelNondim}{(\velocity^{\prime}\nabla^{\prime})\velocity^{\prime}}

\providecommand{\velocityPressureTensorNondim}{\partial_k^{\prime}v^{l \prime}\partial_k^{\prime}v^{l \prime}}

\providecommand{\pressGradNondim}{\nabla^{\prime} p^{\prime}}
\providecommand{\pressLaplacianNondim}{\Delta^{\prime} p^{\prime}}
\providecommand{\laplacianVelNondim}{\Delta^{\prime}\velocity^{\prime}}

\section{Introduction}

The derivation of the equations of the Boussinesq approximation, which describe convective processes, is presented. These equations are nondimensionalized and the physical meaning of the resulting nondimensional parameters is discussed.
 
\section{Derivation of Boussinesq Approximation}

Boussinesq approximation to the equations of convection, which consist of Navier-Stokes equations (cf. chapter \ref{navier_stokes}) and heat equation (cf. chapter \ref{heat}), is based on an assumption that the fluid is incompressible. In addition, the density of the fluid is assumed not to depend on thermodynamic pressure. Furthermore, the temperature variations in the medium are deemed small enough as to accept a linear approximation to the change of density with temperature. 

It must be noted that the variation of density with temperature is only taken into account in the momentum equations while density is constant in other equations. This speaks of the limitations of the approximation: it is only valid for the ``weak convection'' in which the perturbations of density due to temperature gradients are small. Otherwise, the variation of density with temperature must be taken into account in all of the equations, leading to a system different from the one considered here.

The derivations presented below follow the works \cite{convective_stability, boussinesq_validity}.

\subsection{Goal of Boussinesq Approximation}

Boussinesq approximation results in a system of equations that describe the evolution of the \emph{perturbations} of the parameters of the system from the initial reference parameters. For example, the heat transfer equation, in the context of Boussinesq approximation, describes the evolution of the perturbation of temperature $\theta^{\prime} = \theta - \theta_0$ from the reference temperature $\theta_0$. 

Thus, after performing the computations, it is important to remember to add the reference parameters to the computed values in order to obtain the actual information about the system: e.g. $\theta = \theta_0 + \theta^{\prime}$ 

For notational convenience, we shall drop the primes in the exposition that follows, keeping in mind that, in this context, $\theta$ describes the \emph{perturbations} of temperature, not the actual value of temperature.

\subsection{Equation of State}

The equation of state that is of interest in the context of Boussinesq approximation is density as a function of temperature and pressure:

$$ \rho = \rho(\theta, p) $$ 

Neglecting the dependence of density on pressure, and considering only linear terms, the equation of state for density for the perturbations of temperature can be written as

\begin{equation} \label{density}
\rho(\theta_0 + \theta) \approx \rho(\theta_0)(1 - \left. \frac{\theta}{\rho(\theta_0)}\frac{\partial \rho}{\partial \theta} \right|_{\theta_0}) = \rho_0(1 - \beta\theta), ~ \rho(\theta_0) = \rho_0
\end{equation}

where 

$$ \beta = \left. \frac{1}{\rho_0}\frac{\partial \rho}{\partial \theta} \right|_{\theta_0}, ~ [\beta] = \frac{1}{\Theta}$$

is the \emph{volumetric coefficient of thermal expansion} that characterizes the decrease of density with temperature. Its name makes sense if we recall the relationship between the specific volume and density:

$$ v = \frac{1}{\rho} \implies \beta = \left. \frac{1}{\rho_0}\frac{\partial \rho}{\partial \theta} \right|_{\theta_0} = \left. v_0 \frac{\partial v^{-1}}{\partial \theta} \right|_{\theta_0} = \left. -\frac{v_0}{v_0^2} \frac{\partial v}{\partial \theta} \right|_{\theta_0} = \left. -\frac{1}{v_0} \frac{\partial v}{\partial \theta} \right|_{\theta_0} $$

\subsection{Temperature Advection-Diffusion Equation}

Since the temperature advection-diffusion equation in chapter \ref{heat} is linear with respect to $\theta$ and, therefore, invariant with respect to addition of a constant to the temperature, the equation for the perturbation of temperature in the Boussinesq approximation is the same as the original equation if we, for convenience, drop the primes in $\theta^{\prime}$:

\begin{equation} \label{heat_eq}
\partialTimeTemp + \inertTermTemp = \kappa \laplacianTemp 
\end{equation}

\subsection{Navier-Stokes Equations}

The only change in the Navier-Stokes equations resulting from Boussinesq approximation concerns the dependence of density on temperature expressed by the equation (\ref{density}). Substituting (\ref{density}) into the Navier-Stokes equations in chapter \ref{navier_stokes}, we obtain

\begin{multline} \label{NS1-Poisson_boussinesq}
\rho_0(1 - \beta\theta) (\partialTimeVel + \inertTermVel) = -\rho_0 g(1 - \beta\theta)\surfaceNormal - \pressGrad + \mu\laplacianVel \\ = -\rho_0 g\surfaceNormal + \rho_0 g\beta\theta \surfaceNormal - \pressGrad + \mu \laplacianVel
\end{multline}
\begin{equation} \label{NS2-Poisson_boussinesq}
\divergence \velocity = 0
\end{equation}

The term 

$$ -\rho_0\beta\theta (\partialTimeVel + \inertTermVel)$$

on the left-hand side of (\ref{NS1-Poisson_boussinesq}) can be neglected due to the assumption $ \beta\theta \ll 1$ \cite{convective_stability}.

The mechanical equilibrium and, hence, the hydrostatic pressure is determined by the equation 

$$ \nabla p_{hydrostatic} = \rho_0 g \surfaceNormal$$

Because the hydrostatic pressure is of little interest by itself, since it describes the equilibrium case when the density is constant, it can serve as a reference pressure:

\begin{equation} \label{pressure_reference}
p^{\prime} = p - p_{hydrostatic} \Longrightarrow \nabla p^{\prime} = \nabla p - \nabla p_{hydrostatic} = \nabla p - \rho_0 g \surfaceNormal
\end{equation}

Substituting (\ref{pressure_reference}) into the right hand side of (\ref{NS1-Poisson_boussinesq}) eliminates the term $-\rho_0 g\surfaceNormal$. 

Finally, after the operations described above, (\ref{NS1-Poisson_boussinesq}) is divided by $\rho_0$. Dropping the subscript in the reference density $\rho_0$ for convenience but keeping in mind that $\rho$ still denotes the reference density, the Navier-Stokes equations of the Boussinesq approximation finally become 

\begin{equation} \label{NS1-Poisson}
\partialTimeVel + \inertTermVel = g\beta\theta \surfaceNormal - \frac{1}{\rho} \pressGrad + \nu \laplacianVel
\end{equation}
\begin{equation} \label{NS2-Poisson}
\divergence \velocity = 0
\end{equation}

\subsection{Equations of Boussinesq Approximation}

Summarizing preceding considerations, the full set of equations of Boussinesq approximation in dimensional form is the following:

\begin{equation} 
\partialTimeVel + \inertTermVel = g\beta\theta \surfaceNormal - \frac{1}{\rho} \pressGrad + \nu \laplacianVel
\end{equation}
\begin{equation} 
\divergence \velocity = 0
\end{equation}
\begin{equation} 
\partialTimeTemp + \inertTermTemp = \kappa \laplacianTemp 
\end{equation}

\section{Nondimensionalization}

Following the procedures for nondimensionalization described in chapter \ref{heat} and chapter \ref{navier_stokes}, we obtain the following dimensionless system of equations:

\begin{equation} \label{NS1-inertNondim_boussinesq_ch4}
St~ \partialTimeVel + \inertTermVel = \frac{\Theta\beta}{2Fr} \theta \surfaceNormal - \frac{1}{2 Eu}\pressGrad + \frac{1}{Re}\laplacianVel 
\end{equation}
\begin{equation} \label{NS2-inertNondim_boussinesq_ch4}
\divergence \velocity = 0
\end{equation}
\begin{equation} \label{heat_eq_inertNondim_boussinesq_ch4}
 St~ \partialTimeTemp + \inertTermTemp = \frac{1}{Pe} \laplacianTemp  
\end{equation}

In problems concerning the onset of convection, the fluid is initially at rest, so there is no natural characteristic speed of the flow $V$. Therefore, it must be derived from the characteristic length and time scales:

$$ V = \frac{L}{T} $$

Substituting this relationship into the dimensionless coefficients above, we obtain new dimensionless coefficients.

\subsubsection{Strouhal Number}

$$ St = \frac{L}{VT} = 1 $$

Physically, $St = 1$ means that the stationary and nonstationary components of the flow are in balance.


\subsubsection{Measure of Buyoant Force}

$$ \frac{\Theta\beta}{2Fr} = \frac{g \Theta \beta L}{V^2} = \frac{\rho g\Theta \beta L}{\rho V^2} = \frac{Work_{buoyancy}}{2KE} $$

Physically, the nondimensional coefficient $\frac{g \Theta \beta L}{V^2}$ is a measure of the influence of buoyant force on the velocity field. It can be interpreted as the ratio of the work done by the gravity due to buoyancy, resulting from the thermal expansion of the fluid, to the kinetic energy of the flow.

\subsubsection{Euler Number}

$$ Eu = \frac{\rho V^2}{2P} = = \frac{KE}{Work_{pressure}}$$ 

is a measure of the influence of pressure on the inertial flow. It can be interpreted as the ratio of kinetic energy density and the density of work done by the pressure.

\subsubsection{Reynolds Number}

$$ Re = \frac{LV}{\nu} = \frac{L^2}{\nu T} = \frac{T_{viscosity}}{T} $$

In this context, the Reynolds number measures the influence of viscosity processes on the overall flow.

Another way to interpret the Reynolds number is as the ratio of characteristic kinetic energy to the viscous shear stresses of the flow:

$$ KE \sim \rho V^2,~ P_{viscosity} \sim \frac{\rho \nu V}{L},~ \frac{KE}{P_{viscosity}} = \frac{LV}{\nu} = Re $$

\subsubsection{Peclet Number}

$$ Pe = PrRe = \frac{\nu}{\kappa} \frac{L^2}{\nu T} = \frac{L^2}{\kappa T} = \frac{T_{conduction}}{T} $$

In this context, Peclet number measures the influence of heat conduction processes on the overall flow.

\subsection{Nondimensional Form of Boussinesq Equations}

Considering the new nondimensional parameters,
 the equations (\ref{NS1-inertNondim_boussinesq_ch4}), (\ref{NS2-inertNondim_boussinesq_ch4}), and (\ref{heat_eq_inertNondim_boussinesq_ch4}) become
 
\begin{equation} \label{NS1-inertNondim_boussinesq_vel}
\partialTimeVel + \inertTermVel = \frac{g\Theta \beta T^2}{L} \theta \surfaceNormal - \frac{PT^2}{\rho L^2} \pressGrad + \frac{\nu T}{L^2} \laplacianVel 
\end{equation}
\begin{equation} \label{NS2-inertNondim_boussinesq_vel}
\divergence \velocity = 0
\end{equation}
\begin{equation} \label{heat_eq_inertNondim_boussinesq_vel}
\partialTimeTemp + \inertTermTemp = \frac{\kappa T}{L^2} \laplacianTemp  
\end{equation} 

Further investigation of the equations requires information about the characteristic pressure scale $P$ and the characteristic time scale $T$.


\chapter{Boussinesq Approximation and Nondimensionalization for Mantle Convection}

\providecommand{\norm}[1]{\lVert#1 \rVert}
\providecommand{\R}{\begin{pmatrix} R \\ 0 \end{pmatrix}}
\providecommand{\Q}{\begin{pmatrix} Q_1^{T} \\ Q_2^{T} \end{pmatrix}}
\providecommand{\SVD}{\begin{pmatrix} \Sigma \\ 0 \end{pmatrix}}
\providecommand{\SVDr}{\begin{pmatrix} \Sigma_1 & 0 \\ 0 & 0 \end{pmatrix}}
\providecommand{\V}{\begin{pmatrix} V_1^{T} \\ V_2^{T} \end{pmatrix}}
\providecommand{\U}{\begin{pmatrix} U_1  U_2 \end{pmatrix}}
\providecommand{\Vr}{\begin{pmatrix} v_1^{T} \\ \vdots \\ v_r^{T} \end{pmatrix}}
\providecommand{\Vn}{\begin{pmatrix} v_1^{T} \\ \vdots \\ v_n^{T} \end{pmatrix}}
\providecommand{\Vni}{\begin{pmatrix} v_{1i} \dots v_{ni}\end{pmatrix}}
\providecommand{\B}{\begin{pmatrix} b_1^{T} \\ \vdots \\ b_n^{T} \end{pmatrix}}

% Dimensional

\providecommand{\divergence}{\nabla\cdot}

\providecommand{\velocity}{\mathbf{v}}
\providecommand{\substDerivVel}{\frac{D\velocity}{dt}}
\providecommand{\partialTimeVel}{\partial_t\velocity}
\providecommand{\inertTermVel}{(\velocity\nabla)\velocity}

\providecommand{\velocityPressureTensor}{\partial_kv^l\partial_kv^l}

\providecommand{\surfaceNormal}{\mathbf{e}_z}
\providecommand{\pressGrad}{\nabla p}
\providecommand{\pressLaplacian}{\Delta p}
\providecommand{\laplacianVel}{\Delta\velocity}


\providecommand{\partialTimeTemp}{\partial_t\theta}
\providecommand{\inertTermTemp}{(\velocity\nabla)\theta}
\providecommand{\tempGrad}{\nabla \theta}
\providecommand{\laplacianTemp}{\Delta\theta}

% Nondimensional

\providecommand{\divergenceNondim}{\nabla^{\prime}\cdot}

\providecommand{\velocityNondim}{\mathbf{v^{\prime}}}

\providecommand{\substDerivVelNondim}{\frac{D\velocity^{\prime}}{dt^{\prime}}}
\providecommand{\partialTimeVelNondim}{\partial_{t^{\prime}}\velocity^{\prime}}
\providecommand{\inertTermVelNondim}{(\velocity^{\prime}\nabla^{\prime})\velocity^{\prime}}

\providecommand{\velocityPressureTensorNondim}{\partial_k^{\prime}v^{l \prime}\partial_k^{\prime}v^{l \prime}}

\providecommand{\pressGradNondim}{\nabla^{\prime} p^{\prime}}
\providecommand{\pressLaplacianNondim}{\Delta^{\prime} p^{\prime}}
\providecommand{\laplacianVelNondim}{\Delta^{\prime}\velocity^{\prime}}

\section{Introduction} 

The equations of the Boussinesq approximation are considered in the context of mantle convection. The physical considerations underlying the appropriate choice of time and pressure scales for nondimensionalization of these equation are explained, which is commonly not presented in the literature. The physical meaning of all nondimensional numbers arising after nondimensionalization is discussed. Finally, the equations of mantle convection are obtained in the limit of infinite Prandtl number, which results in a system depending on a single nondimensional parameter, the Rayleigh number. 

\section{Choice of Scales}

In chapter \ref{boussinesq} we obtained nondimensionalized form of the equations of Boussinesq approximation from the perspective of the inertial flow\footnote{The meaning of the ``prespective of nondimensionalization'' is explained in (\ref{perspective})}:

\begin{equation} \label{NS1-inertNondim_boussinesq}
\partialTimeVel + \inertTermVel = \frac{g\Theta \beta T^2}{L} \theta \surfaceNormal - \frac{PT^2}{\rho L^2} \pressGrad + \frac{\nu T}{L^2} \laplacianVel 
\end{equation}
\begin{equation} \label{NS2-inertNondim_boussinesq}
\divergence \velocity = 0
\end{equation}
\begin{equation} \label{heat_eq_inertNondim_boussinesq}
\partialTimeTemp + \inertTermTemp = \frac{\kappa T}{L^2} \laplacianTemp  
\end{equation} 

However, mantle convection is characterized by very low Reynolds number flows, which implies that a more appropriate point of view for nondimensionalization is the point of view of the viscosity term in the equation (\ref{NS1-inertNondim_boussinesq}). In order to transform the equations to conform with the new scaling perspective, we multiply the equation (\ref{NS1-inertNondim_boussinesq}) by 

$$ \frac{L^2}{\nu T} $$

which makes the dimensionless coefficient in front of $\laplacianVel$ a unity.

Thus, the equations (\ref{NS1-inertNondim_boussinesq} - \ref{heat_eq_inertNondim_boussinesq}) become

\begin{equation} \label{NS1_viscNondim_boussinesq}
\frac{L^2}{\nu T} (\partialTimeVel + \inertTermVel) = \frac{g\Theta \beta T L}{\nu} \theta \surfaceNormal - \frac{PT}{\rho \nu} \pressGrad + \laplacianVel 
\end{equation}
\begin{equation} \label{NS2_viscNondim_boussinesq}
\divergence \velocity = 0
\end{equation}
\begin{equation} \label{heat_eq_viscNondim_boussinesq}
\partialTimeTemp + \inertTermTemp = \frac{\kappa T}{L^2} \laplacianTemp  
\end{equation} 

Further investigation of these equations requires additional information about the time and pressure scales, $T$ and $P$.

\subsection{Time Scale}

There are two natural choices of scales in the context of the highly viscous flows that are characteristic of mantle convection. \cite{getling}

The first choice is based on the viscosity time scale:

$$ T_{viscosity} = \frac{L^2}{\nu} $$

The second choice is based on the heat conduction time scale: 

$$ T_{conduction} = \frac{L^2}{\kappa} $$

In order to choose the appropriate scale for the context of mantle convection, we observe that their ratio is equal to the Prandtl number:

$$ \frac{T_{conduction}}{T_{viscosity}} = \frac{\nu}{\kappa} = Pr $$ 

A commonly used approximation for the dynamics of Earth's mantle is the limit of infinite viscosity, which is equivalent to the limit of infinite Prandtl number since thermal conductivity is usually fixed or bounded:

$$ \nu \longrightarrow \infty \Longleftrightarrow Pr \longrightarrow \infty$$

Physically, this limit can be interpreted as

$$ T_{viscosity} \ll T_{conduction} $$

or 

$$ T_{viscosity} \longrightarrow 0 $$

Physically, this limit means that information due to viscous interactions propagates infinitely fast, and infinitely fast processes cannot serve as a measure of other processes because they are infinitely slow relative to infinitely fast processes. Thus, the viscosity time scale becomes singular in the common mantle convection approximation, which makes it an inappropriate choice for the time scale.

Therefore, the only choice left is the time scale associated with heat conduction,

$$ T = T_{conduction} = \frac{L^2}{\kappa} $$

Another argument supporting this choice of scale is the fact that heat conduction takes place regardless of the macroscopic flow of the fluid. Since initially the fluid is at rest in the common problems of mantle convection, measurable viscous dissipation processes cannot start until the fluid has been set in an adequately intense macroscopic flow. At the same time, heat conduction starts immediately because a gradient of temperature is always present in the mantle.

\subsection{Pressure Scale}

In chapter \ref{stokes} we justified an appropriate pressure scale for low Reynolds number flows based on an estimate of viscous shear stress:

$$ P = \frac{\rho \nu V}{L} = \frac{\rho \nu}{T} $$

Following \cite[pp. 433-434]{leal}, we shall verify that this choice of scale is appropriate by comparing it with another possible choice of scale based on kinetic energy.

Given $9$ parameters describing the convection ($\mathbf{x}, \velocity, \rho, \nu, \kappa, g, p, \theta, t$) and $4$ units of measurement (length, time, mass, temperature), by Buckingham's $\Pi$-theorem we can form $9 - 4 = 5$ nondimensional parameters describing the system. \cite{barenblatt1} (cf. also \nameref{appendix_pi})

From (\ref{NS1_viscNondim_boussinesq}), (\ref{NS2_viscNondim_boussinesq}), and (\ref{heat_eq_viscNondim_boussinesq}) we see that they are the following:

$$ Pr = \frac{\nu}{\kappa} $$
$$ \frac{PT}{\rho \nu} $$
$$  \frac{\rho L^2}{PT^2}$$ 
$$ Ra = \frac{g\Theta \beta L^3}{\nu\kappa} $$
$$ \frac{\kappa T}{L^2} $$

From this we can see that there are two choices of nondimensionalizing the pressure:

$$ P_1 = \frac{\rho \nu}{T} $$
$$ P_2 = \frac{\rho L^2}{T^2}$$ 

Their ratio is :

$$ \frac{P_1}{P_2} = \frac{\nu T}{L^2} = \frac{T}{T_{viscosity}} $$

Given our choice of time scale, we find that the ratio is equal to the Prandtl number:

$$ T = T_{conduction} = \frac{L^2}{\kappa} $$
$$ \frac{P_1}{P_2} = \frac{T_{conduction}}{T_{viscosity}} = \frac{\nu}{\kappa} = Pr \Longleftrightarrow P_2 = \frac{1}{Pr} P_1 $$

Considering these choices of scale, the coefficient multiplying the gradient of pressure in the equation (\ref{NS1_viscNondim_boussinesq}) becomes either 

$$ \frac{P_1T}{\rho \nu} = 1$$

or 

$$ \frac{P_2T}{\rho \nu} = \frac{1}{Pr}\frac{P_1T}{\rho \nu} = \frac{1}{Pr}$$

In the limit $ Pr \longrightarrow \infty$ the scale associated with $P_2$ implies that the gradient of pressure term in the equation (\ref{NS1_viscNondim_boussinesq}) will vanish. This contradicts the physical fact that pressure influences the flow even at low Reynolds numbers.\cite[pp. 433-434]{leal}

Thus, the appropriate choice of characteristic pressure scale is, as in chapter \ref{stokes}, associated with viscous shear stress:

$$ P = P_1 = \frac{\rho \nu}{T} = \frac{\rho \nu \kappa}{L^2}$$

\section{Nondimensional Form of Equations}

Considering the choices of scales discussed above, the equations (\ref{NS1_viscNondim_boussinesq}), (\ref{NS2_viscNondim_boussinesq}), and (\ref{heat_eq_viscNondim_boussinesq}) finally become

\begin{equation} \label{NS1_viscNondim_boussinesq_time}
\frac{1}{Pr} (\partialTimeVel + \inertTermVel) = Ra ~\theta \surfaceNormal - \pressGrad + \laplacianVel 
\end{equation}
\begin{equation} \label{NS2_viscNondim_boussinesq_time}
\divergence \velocity = 0
\end{equation}
\begin{equation} \label{heat_eq_viscNondim_boussinesq_time}
\partialTimeTemp + \inertTermTemp = \laplacianTemp  
\end{equation} 

The only new nondimensional parameter is the Rayleigh number $Ra$.

\subsubsection{Rayleigh Number}

Rayleigh number 

$$ Ra = \frac{g\Theta \beta L^3}{\nu\kappa} = \frac{g\Theta \beta L T}{\nu} = \frac{g\Theta \beta T}{\frac{\nu}{L}} = \frac{\rho V_{buoyancy}}{\rho V_{viscosity}} $$

is a measure of the influence of temperature on the mechanical motion of the fluid. 

It can be interpreted as a ratio of momentum gained due to the buoyant force, resulting from thermal expansion of the fluid, and the momentum dissipated by the viscosity. 

Another interpretation of Rayleigh number is a ratio of the work of buoyant forces and viscous shear stress:

$$ Ra = \frac{g\Theta \beta L^3}{\nu\kappa} = \frac{\rho g\Theta \beta L}{\frac{\rho \nu}{T}} = \frac{E_{buoyancy}}{P_{viscosity}} $$

In other words, Rayleigh number measures how far from the equilibrium a system is: indeed, while the buoyant force destabilizes the motion, viscosity stabilizes it through energy dissipation.

\subsection{Limit of Infinite Prandtl Number}

A common approximation in mantle convection problems is the limit of infinite Prandtl number. By taking this limit in the equations (\ref{NS1_viscNondim_boussinesq_time}), (\ref{NS2_viscNondim_boussinesq_time}), and (\ref{heat_eq_viscNondim_boussinesq_time}), we obtain the following system of equations in nondimensional form:

\begin{equation} \label{NS1_viscNondim_boussinesq_prandtl}
\laplacianVel = \pressGrad - Ra ~\theta \surfaceNormal
\end{equation}
\begin{equation} \label{NS2_viscNondim_boussinesq_prandtl}
\divergence \velocity = 0
\end{equation}
\begin{equation} \label{heat_eq_viscNondim_boussinesq_prandtl}
\partialTimeTemp + \inertTermTemp = \laplacianTemp  
\end{equation} 

As we can see, the only parameter defining the system is the Rayleigh number, which makes it especially suitable for numerical experiments.


% \appendix
%    Include appendix "chapters" here.
% \chapter{Buckingham's $\Pi$-Theorem} \label{appendix_pi}
% \providecommand{\norm}[1]{\lVert#1 \rVert}
\providecommand{\R}{\begin{pmatrix} R \\ 0 \end{pmatrix}}
\providecommand{\Q}{\begin{pmatrix} Q_1^{T} \\ Q_2^{T} \end{pmatrix}}
\providecommand{\SVD}{\begin{pmatrix} \Sigma \\ 0 \end{pmatrix}}
\providecommand{\SVDr}{\begin{pmatrix} \Sigma_1 & 0 \\ 0 & 0 \end{pmatrix}}
\providecommand{\V}{\begin{pmatrix} V_1^{T} \\ V_2^{T} \end{pmatrix}}
\providecommand{\U}{\begin{pmatrix} U_1  U_2 \end{pmatrix}}
\providecommand{\Vr}{\begin{pmatrix} v_1^{T} \\ \vdots \\ v_r^{T} \end{pmatrix}}
\providecommand{\Vn}{\begin{pmatrix} v_1^{T} \\ \vdots \\ v_n^{T} \end{pmatrix}}
\providecommand{\Vni}{\begin{pmatrix} v_{1i} \dots v_{ni}\end{pmatrix}}
\providecommand{\B}{\begin{pmatrix} b_1^{T} \\ \vdots \\ b_n^{T} \end{pmatrix}}

\providecommand{\ort}{\begin{pmatrix} 0 \\ \vdots \\ 1 \\ \vdots \\ 0 \end{pmatrix}}

\providecommand{\X}{\begin{pmatrix} [X_1] & \hdots & [X_n] \end{pmatrix}}

% Dimensional

\providecommand{\divergence}{\nabla\cdot}

\providecommand{\velocity}{\mathbf{v}}
\providecommand{\substDerivVel}{\frac{D\velocity}{dt}}
\providecommand{\partialTimeVel}{\partial_t\velocity}
\providecommand{\inertTermVel}{(\velocity\nabla)\velocity}

\providecommand{\velocityPressureTensor}{\partial_kv^l\partial_kv^l}

\providecommand{\surfaceNormal}{\mathbf{e}_z}
\providecommand{\pressGrad}{\nabla p}
\providecommand{\pressLaplacian}{\Delta p}
\providecommand{\laplacianVel}{\Delta\velocity}


\providecommand{\partialTimeTemp}{\partial_t\theta}
\providecommand{\inertTermTemp}{(\velocity\nabla)\theta}
\providecommand{\tempGrad}{\nabla \theta}
\providecommand{\laplacianTemp}{\Delta\theta}

% Nondimensional

\providecommand{\divergenceNondim}{\nabla^{\prime}\cdot}

\providecommand{\velocityNondim}{\mathbf{v^{\prime}}}

\providecommand{\substDerivVelNondim}{\frac{D\velocity^{\prime}}{dt^{\prime}}}
\providecommand{\partialTimeVelNondim}{\partial_{t^{\prime}}\velocity^{\prime}}
\providecommand{\inertTermVelNondim}{(\velocity^{\prime}\nabla^{\prime})\velocity^{\prime}}

\providecommand{\velocityPressureTensorNondim}{\partial_k^{\prime}v^{l \prime}\partial_k^{\prime}v^{l \prime}}

\providecommand{\pressGradNondim}{\nabla^{\prime} p^{\prime}}
\providecommand{\pressLaplacianNondim}{\Delta^{\prime} p^{\prime}}
\providecommand{\laplacianVelNondim}{\Delta^{\prime}\velocity^{\prime}}



% \input{structure.tex} % Include the structure.tex file which specified the document structure and layout

\hyphenation{Fortran hy-phen-ation} % Specify custom hyphenation points in words with dashes where you would like hyphenation to occur, or alternatively, don't put any dashes in a word to stop hyphenation altogether

%----------------------------------------------------------------------------------------
%	HEADERS
%----------------------------------------------------------------------------------------

% \renewcommand{\sectionmark}[1]{\markright{\spacedlowsmallcaps{#1}}} % The header for all pages (oneside) or for even pages (twoside)
%\renewcommand{\subsectionmark}[1]{\markright{\thesubsection~#1}} % Uncomment when using the twoside option - this modifies the header on odd pages
% \lehead{\mbox{\llap{\small\thepage\kern1em\color{halfgray} \vline}\color{halfgray}\hspace{0.5em}\rightmark\hfil}} % The header style

% \pagestyle{scrheadings} % Enable the headers specified in this block

%----------------------------------------------------------------------------------------
%	TABLE OF CONTENTS & LISTS OF FIGURES AND TABLES
%----------------------------------------------------------------------------------------

\section{Introduction}
 
The Buckingham $\Pi$-Theorem is discussed both from the perspective of revealing the rank of a set of physical quantities, as well as from the perspective of reducing the number of parameters in an equation linking physical quantities. It is shown that the Buckingham $\Pi$-Theorem is simply a consequence of a fundamental theorem of linear algebra relating the dimensions of the image, kernel, and the number of columns of a matrix.   
% 
% 
% \setcounter{tocdepth}{2} % Set the depth of the table of contents to show sections and subsections only
% 
% \tableofcontents % Print the table of contents
% 
% \newpage % Start the article content on the second page, remove this if you have a longer abstract that goes onto the second page

\subsection{Content of Buckingham's $\Pi$-Theorem}

The content of Buckingham's $\Pi$-theorem is an answer to the question:
\begin{quote}
Why out of $n$ physical quantities, of which $k$ are dimensionally independent, can we obtain $n-k$ dimensionless parameters?
\end{quote}
 
The answer is simply a consequence of a fundamental theorem of linear algebra. \cite{zorich, sonin, bluman_kumei, olsen}

\section{Dimensionality Reduction of a System of Physical Quantities}

\subsection{Physical Dimensions Form a Linear Space}

Physical quantities are measured against a conventionally chosen system of units $\mathbf{X} = \X$. Therefore, the dimension of any physical quantity is simply an algebraic combination of the units of measurements: 

$$[A] = [ ~\prod \limits_{i=1}^{n} X_i^{d_A^i}~ ] = \sum \limits_{i=1}^n d_A^i [X_i]$$ 

For example, in the unit system of mass $M$, length $L$, and time $T$, the dimension of $P$ pressure is 

$$[P] = \frac{M}{LT^2} = [M] - [L] - 2 [T] $$

We see that we can establish an isomorphism between the dimensions of physical quantities and a linear space of vectors corresponding to these dimensions. 

Indeed, consider the dimension of a product of two physical quantities:
$$[AB] = [~ \prod \limits_{i=1}^{n} X_i^{d_A^i} \prod \limits_{i=1}^{n} X_i^{d_B^i} ~] =  [~ \prod \limits_{i=1}^{n} X_i^{d_A^i}X_i^{d_B^i} ~] = [~ \prod \limits_{i=1}^{n} X_i^{d_A^i + d_B^i} ~] = \sum \limits_{i=1}^n (d_A^i + d_B^i) [X_i] = [A] + [B] $$

Evidently, vectors $[X_i]$ form a basis of this linear space of dimensions. It is convenient to choose a canonical orthonormal basis for the fundamental units of measurements:

$$ [X_i] \equiv \ort \text{where $1$ is at the $i$-th component and $0$ elsewhere.} $$

For example, if we establish the following correspondence between the basis vectors and the dimensions of mass $M$, length $L$, and time $T$:

$$ [M] =  \begin{pmatrix} 1 \\ 0 \\ 0 \end{pmatrix},~ [L] =  \begin{pmatrix} 0 \\ 1 \\ 0 \end{pmatrix},~ [T] =  \begin{pmatrix} 0 \\ 0 \\ 1 \end{pmatrix}$$

the dimension of pressure will be expressed thus:

$$[P] = \frac{M}{LT^2} = [M] - [L] - 2 [T] = \begin{pmatrix} 1 \\ -1 \\ -2 \end{pmatrix} $$

\subsection{Dimensional Matrix}

Considering these facts and recalling the definition of matrix-vector multiplication, we can see that $\mathbf{X} = \X$ forms a \emph{dimensional matrix} of the system of physical quantities. Thus, the dimension vector $d_A$ of any physical quantity $[A]$ in the basis $\mathbf{X} = \X$ is linearly related to the dimensional matrix $\mathbf{X}$:

$$ [A] = \sum \limits_{i=1}^n d_A^i [X_i] = \mathbf{X}d_A$$

\subsubsection{Dimensionless Quantities and Kernel of the Dimensional Matrix}

A physical quantity $[A]$ is \emph{dimensionless} if 

$$A = \prod \limits_{i=1}^{n} X_i^{d_A^i} = 1 $$ 

which means that 

$$[A] = \sum \limits_{i=1}^n d_A^i [X_i] = \mathbf{X}d_A = [1] = 0 \Longrightarrow d_A \in \ker \mathbf{X}$$ 

Thus, there is a one-to-one correspondence between the kernel, or nullspace, of the dimensional matrix $\mathbf{X}$ and the nondimensional quantities that can be formed out of the system of physical quantities $\mathbf{X} = \X$. Therefore, the number of distinct nondimensional numbers that it is possible to produce out of a system of dimensional quantities $(X_1, \hdots, X_n)$ is equal to $\text{dim ker}~ \mathbf{X}$, i.e. the dimension of the nullspace of the dimension matrix $\mathbf{X}$.

\subsubsection{Dimensionally Independent Physical Quantities and the Rank of the Dimensional Matrix}

Physical quantities $(X_1, \hdots, X_n)$ are dimensionally independent if their only combination that can produce a dimensionless number is a trivial combination. In other words, the dimension vectors of dimensionally independent quantities are linearly independent:

$$ \prod \limits_{i=1}^{n} X_i^{d_A^i} = 1 \Longleftrightarrow \sum \limits_{i=1}^n d^i [X_i] = 0 \Longrightarrow d^i = 0 $$

Thus, the dimension matrix composed of the dimensions of $n$ dimensionally independent physical quantities has full rank:

$$ \text{rank}~ \mathbf{X} = n $$ 

\subsection{Buckingham's Theorem}

Now it is possible to prove Buckingham's $\Pi$ \-theorem. Usually, the number of dimensionally independent physical quantities in the system is known, and it is interesting to know how many dimensionless quantities can be produced out of this system. Such question is of interest since it is preferable  to work with as little number of parameters as possible.

Let $(X_1, \hdots, X_k, \hdots, X_n)$ be a system of $n$ physical quantities of which the first $k$ are dimensionally independent. The dimensional matrix of this system 

$$\mathbf{X} = ([X_1], \hdots, [X_k], \hdots, [X_n])$$

has $k$ linearly independent columns and, hence, its rank is $k$:

$$ \text{dim im} ~\mathbf{X} = \text{rank}~ \mathbf{X} = k$$ 

By a fundamental theorem of linear algebra, the sum of the dimensions of the image and the kernel of a linear operator is equal to the number of columns of its matrix:

$$ \text{dim ker}~ \mathbf{X} + \text{dim im}~ \mathbf{X} = \text{dim ker}~ \mathbf{X} + k =  n $$ 

Since it was established that the number of distinct nondimensional numbers characterizing the system $(X_1, \hdots, X_k, \hdots, X_n)$ is equal to the dimension of the nullspace of its dimensional matrix, we obtain the result of the $\Pi$-theorem:

$$ \text{dim ker}~ \mathbf{X} =  n - k $$ 

Thus, given $n$ physical quantities of which $k$ are dimensionally independent, we can form $n - k$ nondimensional numbers fully characterizing the system. This allows us to reduce the number of parameters needed to describe the system.

In addition, we established that, given a dimensional matrix of a set of physical quantities, the problem of finding the number of dimensionally independent quantities among them reduces to the problem of computing the rank of the dimensional matrix.

\section{Dimensionality Reduction of a Physical Functional Dependence}

We have now established that given $n$ physical quantities of which $k$ are dimensionally independent, we can form $n - k$ nondimensional numbers fully characterizing the system. However, what does this imply for functional dependencies of one physical quantity on others? 

Based on the previous result we intuitively expect that a functional dependence of the form

$$X_0 = f(X_1, \hdots, X_k, \hdots, X_n)$$

could be described by $n - k$ nondimensional parameters instead of $n$ original dimensional parameters.

Indeed, let $(X_0, X_1, \hdots, X_k, \hdots, X_n)$ be a system of physical quantities of which $k$ are dimensionally independent: the dimensionally independent quantities will be used as the units of measurement. 
\newpage
Furthermore, $X_0$ is a function of other quantities:

\begin{equation} \label{1}
X_0 = f(X_1, \hdots, X_k, \hdots, X_n)
\end{equation}

The dimensions of $X_0$ and $(X_{k+1}, \hdots, X_n)$ can be expressed in terms of the dimensions of the dimensionally independent quantities $(X_1, \hdots, X_k)$, since they are our units of measurement:

$$ [X_0] = \sum \limits_{i=1}^{k} d_0^i [X_i] = [~\prod \limits_{i=1}^{k} X_i^{d_0^i}~] $$

$$ [X_{k+j}] = \sum \limits_{i=1}^{k} d_j^i [X_i] = [~\prod \limits_{i=1}^{k} X_i^{d_j^i}~],~ j \in \overline{1, n-k} $$

If we change the scales for the units of measurements, i.e.

\begin{equation} \label{scales}
X_i \mapsto a_i X_i,~ i \in \overline{1,k} 
\end{equation}

the dimensionally-dependent quantities must transform accordingly:

\begin{equation} \label{scales1}
 X_0 \mapsto \prod \limits_{i=1}^{k} (a_iX_i)^{d_0^i} = (\prod \limits_{i=1}^{k} a_i^{d_0^i}) \prod \limits_{i=1}^{k} X_i^{d_0^i} = (\prod \limits_{i=1}^{k} a_i^{d_0^i}) X_0 
\end{equation}

\begin{equation} \label{scales2}
 X_{k+j} \mapsto \prod \limits_{i=1}^{k} (a_iX_i)^{d_j^i} = (\prod \limits_{i=1}^{k} a_i^{d_j^i}) \prod \limits_{i=1}^{k} X_i^{d_j^i} = (\prod \limits_{i=1}^{k} a_i^{d_j^i}) X_{k+j},~ j \in \overline{1,n-k}
\end{equation}

Since $(X_1, \hdots, X_k)$ are dimensionally independent units of measurement, it is possible to choose scales based on these quantities, i.e.

$$ a_i = \frac{1}{X_i},~ \in \overline{1,k} $$

In such case, (\ref{scales}), (\ref{scales1}), and (\ref{scales2}) become

\begin{equation} \label{scales_pi}
a_i X_i = 1,~ i \in \overline{1,k} 
\end{equation}

\begin{equation} \label{scales1_pi}
(\prod \limits_{i=1}^{k} a_i^{d_0^i}) X_0 = \frac{X_0}{(\prod \limits_{i=1}^{k} X_i^{d_0^i})} = \Pi
\end{equation}

\begin{equation} \label{scales2_pi}
(\prod \limits_{i=1}^{k} a_i^{d_j^i}) X_{k+j} = \frac{X_{k+j}}{(\prod \limits_{i=1}^{k} X_i^{d_j^i})} = \Pi_j,~ j \in \overline{1,n-k}
\end{equation}

Consequently, the physical functional dependence (\ref{1}) transforms accordingly under the change of scales (\ref{scales_pi}), (\ref{scales1_pi}), and (\ref{scales2_pi}):

$$ X_0 = f(X_1, \hdots, X_k, \hdots, X_n) \mapsto \Pi = f(1, \hdots, 1, \Pi_1, \hdots, \Pi_{n-k}) = F(\Pi_1, \hdots, \Pi_{n-k})$$

Thus, a dimensional functional dependence involving $n$ dimensional parameters

$$X_0 = f(X_1, \hdots, X_k, \hdots, X_n)$$

has been reduced to a nondimensional functional dependence involving only $n-k$ dimensionless parameters

$$ \Pi = F(\Pi_1, \hdots, \Pi_{n-k}) $$

This relationship is particularly useful, for example, in nondimensionalization of fluid dynamics equations because it reveals the relative influence of various terms of the equations on the overall behavior of the flow.


% \newpage
% 
% \bibliographystyle{plain}
% \bibliography{../../bibliography}

% \chapter{Poisson Equation for Pressure} \label{appendix_poisson}
% \providecommand{\norm}[1]{\lVert#1 \rVert}
\providecommand{\R}{\begin{pmatrix} R \\ 0 \end{pmatrix}}
\providecommand{\Q}{\begin{pmatrix} Q_1^{T} \\ Q_2^{T} \end{pmatrix}}
\providecommand{\SVD}{\begin{pmatrix} \Sigma \\ 0 \end{pmatrix}}
\providecommand{\SVDr}{\begin{pmatrix} \Sigma_1 & 0 \\ 0 & 0 \end{pmatrix}}
\providecommand{\V}{\begin{pmatrix} V_1^{T} \\ V_2^{T} \end{pmatrix}}
\providecommand{\U}{\begin{pmatrix} U_1  U_2 \end{pmatrix}}
\providecommand{\Vr}{\begin{pmatrix} v_1^{T} \\ \vdots \\ v_r^{T} \end{pmatrix}}
\providecommand{\Vn}{\begin{pmatrix} v_1^{T} \\ \vdots \\ v_n^{T} \end{pmatrix}}
\providecommand{\Vni}{\begin{pmatrix} v_{1i} \dots v_{ni}\end{pmatrix}}
\providecommand{\B}{\begin{pmatrix} b_1^{T} \\ \vdots \\ b_n^{T} \end{pmatrix}}

\providecommand{\ort}{\begin{pmatrix} 0 \\ \vdots \\ 1 \\ \vdots \\ 0 \end{pmatrix}}

\providecommand{\X}{\begin{pmatrix} [X_1] & \hdots & [X_n] \end{pmatrix}}

% Dimensional

\providecommand{\divergence}{\nabla\cdot}

\providecommand{\velocity}{\mathbf{v}}
\providecommand{\substDerivVel}{\frac{D\velocity}{dt}}
\providecommand{\partialTimeVel}{\partial_t\velocity}
\providecommand{\inertTermVel}{(\velocity\nabla)\velocity}

\providecommand{\velocityPressureTensor}{\partial_kv^l\partial_kv^l}

\providecommand{\surfaceNormal}{\mathbf{e}_z}
\providecommand{\pressGrad}{\nabla p}
\providecommand{\pressLaplacian}{\Delta p}
\providecommand{\laplacianVel}{\Delta\velocity}


\providecommand{\partialTimeTemp}{\partial_t\theta}
\providecommand{\inertTermTemp}{(\velocity\nabla)\theta}
\providecommand{\tempGrad}{\nabla \theta}
\providecommand{\laplacianTemp}{\Delta\theta}

% Nondimensional

\providecommand{\divergenceNondim}{\nabla^{\prime}\cdot}

\providecommand{\velocityNondim}{\mathbf{v^{\prime}}}

\providecommand{\substDerivVelNondim}{\frac{D\velocity^{\prime}}{dt^{\prime}}}
\providecommand{\partialTimeVelNondim}{\partial_{t^{\prime}}\velocity^{\prime}}
\providecommand{\inertTermVelNondim}{(\velocity^{\prime}\nabla^{\prime})\velocity^{\prime}}

\providecommand{\velocityPressureTensorNondim}{\partial_k^{\prime}v^{l \prime}\partial_k^{\prime}v^{l \prime}}

\providecommand{\pressGradNondim}{\nabla^{\prime} p^{\prime}}
\providecommand{\pressLaplacianNondim}{\Delta^{\prime} p^{\prime}}
\providecommand{\laplacianVelNondim}{\Delta^{\prime}\velocity^{\prime}}



% \input{structure.tex} % Include the structure.tex file which specified the document structure and layout

\hyphenation{Fortran hy-phen-ation} % Specify custom hyphenation points in words with dashes where you would like hyphenation to occur, or alternatively, don't put any dashes in a word to stop hyphenation altogether

%----------------------------------------------------------------------------------------
%	HEADERS
%----------------------------------------------------------------------------------------

% \renewcommand{\sectionmark}[1]{\markright{\spacedlowsmallcaps{#1}}} % The header for all pages (oneside) or for even pages (twoside)
%\renewcommand{\subsectionmark}[1]{\markright{\thesubsection~#1}} % Uncomment when using the twoside option - this modifies the header on odd pages
% \lehead{\mbox{\llap{\small\thepage\kern1em\color{halfgray} \vline}\color{halfgray}\hspace{0.5em}\rightmark\hfil}} % The header style

% \pagestyle{scrheadings} % Enable the headers specified in this block

%----------------------------------------------------------------------------------------
%	TABLE OF CONTENTS & LISTS OF FIGURES AND TABLES
%----------------------------------------------------------------------------------------

\section{Introduction}
 
Basic tensor notation is introduced and is used to derive the Poisson equation for pressure for the incompressible flow. This Poisson equation relates pressure and velocity and thus shows that in incompressible flows the pressure is completely determined by the velocity field.
 
% 
% 
% \setcounter{tocdepth}{2} % Set the depth of the table of contents to show sections and subsections only
% 
% \tableofcontents % Print the table of contents
% 
% \newpage % Start the article content on the second page, remove this if you have a longer abstract that goes onto the second page

\section{Tensor Notation}

As will be seen later, tensor notation is convenient for writing the equations of fluid dynamics. The purpose of this notation is 
to express the sums of large number of terms compactly, which is achieved by implicit summation over repeating indices.

For our purposes, we establish the following conventions: 
\begin{enumerate}
 \item Spatial coordinates $xyz$ are denoted by numbers from $1$ to $3$
 \item $x^i$ is $i$-th component of a vector $\mathbf{x}$, $\partial_i \equiv \frac{\partial}{\partial x^i}$
 \item Summation is implicit over indices repeated within the same expression (e.g. $a_ib_i = \sum_{i=1}^{3}a_ib_i$)
\end{enumerate}

For example, matrix-vector multiplication is expressed elegantly in tensor notation: $$(A\mathbf{x})^i = \sum_{j=1}^{n}a_{ij}x^j \Longleftrightarrow A\mathbf{x} \equiv a_{ij}x^j$$

In this case, $i$ is called \emph{running index} since it denotes the component of the resulting vector that can be chosen freely by us, whereas the index $j$ is called \emph{dummy index}, since it is only used in the summation and has no other significance. 

To further illustrate the convenience of this notation, consider an example of the divergence of a vector field: $$\nabla \cdot \velocity = \partial_xv^{x} + \partial_yv^{y} + \partial_zv^{z} = \sum_{k=1}^{3}\partial_kv^k \equiv \partial_kv^{k}$$
 
\subsection{Navier-Stokes Equations in Tensor Notation}
 
Considering previous explanation of tensor notation, the Navier-Stokes equations for incompressible flow can be written thus:
 
\begin{equation} \label{NS1-Tensor}
\partial_tv^i + v^k\partial_kv^i = -g\delta_{i3} - \frac{1}{\rho}\partial_ip + \nu \partial_k^2 v^i  
\end{equation}
\begin{equation} \label{NS2-Tensor}
\partial_kv^{k} = 0  
\end{equation}

where $\delta_{ij}$ is \emph{Kronecker's delta} function, which is $0$ except when $i=j$: then it is equal to $1$.

\section{Poisson Equation for Pressure Field in an Incompressible Flow}

When the flow is incompressible, the pressure field is determined completely by the velocity field of the flow. This can be shown with relative ease by using the tensor notation discussed above.

Indeed, let's apply the divergence operator to equation (\ref{NS1}) term by term:

\begin{enumerate}
 \item $\divergence \partial_t\velocity = \partial_t \ (\divergence \velocity) = 0$
 \item $\divergence [\inertTermVel] = \partial_l(\inertTermVel)^l = \partial_l(v^k\partial_kv^l) = \partial_lv^k\partial_kv^l + v^k\partial_l\partial_kv^l$ \\Since $v^k\partial_l\partial_kv^l = v^k\partial_k(\partial_lv^l) = v^k\partial_k (0) = 0 \implies \divergence [\inertTermVel] = \partial_lv^k\partial_kv^l$
 \item $\divergence g\surfaceNormal = 0$\footnote{This is true in general for any potential field when no sources are present inside the domain.}
 \item $\divergence\frac{1}{\rho}\pressGrad = \frac{1}{\rho}\Delta p$
 \item $\divergence (\nu\Delta\velocity) = \nu\partial_l(\Delta\velocity)^l = \nu\partial_l\partial_k\partial_kv^l = \nu\partial_k\partial_k(\partial_lv^l) = 0$
\end{enumerate}

Thus, only terms $2$ and $4$ remain in (\ref{NS1}), thus yielding the \emph{Poisson equation} for pressure:

\begin{equation}
\Delta p = - \rho ~\partial_lv^k\partial_kv^l
\end{equation}

This means that equations (\ref{NS1}) and (\ref{NS2}) can be written in an equivalent form, using tensor notation, which relates pressure and velocity fields more directly:

\begin{equation} \label{NS1-Poisson}
\partialTimeVel + \inertTermVel = -g\surfaceNormal - \frac{1}{\rho}\pressGrad + \nu\laplacianVel 
\end{equation}
\begin{equation} \label{NS2-Poisson}
\Delta p = - \rho ~\velocityPressureTensor
\end{equation}

The utility of tensor notation is evident: without Einstein's convention, equation (\ref{NS2-Poisson}) has $6$ terms, even after complete simplification, versus one term in its compact tensor representation. Furthermore, tensor notation allowed us to compute complicated quantities such as $\divergence [\inertTermVel]$ with just a few steps and basic knowledge of algebraic properties of finite sums and differentiation.

Finally, equation (\ref{NS2-Poisson}) is useful for the dimensional and scaling analysis of the Navier-Stokes equations. Indeed, the fact that a velocity field has zero divergence is scale-invariant since the equation $\divergence \velocity = 0$ will hold under any choice of scale. This means that this equation does not provide any additional information about the behavior of the system under various choices of scales. On the other hand, the coefficients in the equation (\ref{NS2-Poisson}) vary under different scales. Therefore, equation (\ref{NS2-Poisson}) can reveal the relative importance of pressure and velocity fields under various choices of scale. 

\section{Laplace Equation for Pressure in Stokes Equations} \label{pressure_laplace}

Considering the estimations of the nondimensional parameters presented in (\ref{stokes_parameters}), the equations (\ref{NS1-viscNondim}) and (\ref{NS2-viscNondim}) become:

\begin{equation} 
Re(\partialTimeVel + \inertTermVel) = -\frac{gL^2}{\nu V} \surfaceNormal - \pressGrad + \laplacianVel 
\end{equation}
\begin{equation} 
\pressLaplacian = Re ~\velocityPressureTensor
\end{equation}

\backmatter

%    Bibliography styles amsplain or harvard are also acceptable.
\bibliographystyle{amsplain}
\bibliography{bibliography}
%    See note above about multiple indexes.
% \printindex

\end{document}

%-----------------------------------------------------------------------
% End of amsbook-template.tex
%-----------------------------------------------------------------------
