\providecommand{\norm}[1]{\lVert#1 \rVert}
\providecommand{\R}{\begin{pmatrix} R \\ 0 \end{pmatrix}}
\providecommand{\Q}{\begin{pmatrix} Q_1^{T} \\ Q_2^{T} \end{pmatrix}}
\providecommand{\SVD}{\begin{pmatrix} \Sigma \\ 0 \end{pmatrix}}
\providecommand{\SVDr}{\begin{pmatrix} \Sigma_1 & 0 \\ 0 & 0 \end{pmatrix}}
\providecommand{\V}{\begin{pmatrix} V_1^{T} \\ V_2^{T} \end{pmatrix}}
\providecommand{\U}{\begin{pmatrix} U_1  U_2 \end{pmatrix}}
\providecommand{\Vr}{\begin{pmatrix} v_1^{T} \\ \vdots \\ v_r^{T} \end{pmatrix}}
\providecommand{\Vn}{\begin{pmatrix} v_1^{T} \\ \vdots \\ v_n^{T} \end{pmatrix}}
\providecommand{\Vni}{\begin{pmatrix} v_{1i} \dots v_{ni}\end{pmatrix}}
\providecommand{\B}{\begin{pmatrix} b_1^{T} \\ \vdots \\ b_n^{T} \end{pmatrix}}

% Dimensional

\providecommand{\divergence}{\nabla\cdot}

\providecommand{\velocity}{\mathbf{v}}
\providecommand{\substDerivVel}{\frac{D\velocity}{dt}}
\providecommand{\partialTimeVel}{\partial_t\velocity}
\providecommand{\inertTermVel}{\velocity\cdot\nabla \velocity}

\providecommand{\velocityPressureTensor}{\partial_kv^l\partial_kv^l}

\providecommand{\surfaceNormal}{\mathbf{e}_z}
\providecommand{\pressGrad}{\nabla p}
\providecommand{\pressLaplacian}{\Delta p}
\providecommand{\laplacianVel}{\Delta\velocity}


\providecommand{\partialTimeTemp}{\partial_t\theta}
\providecommand{\inertTermTemp}{\velocity\cdot\nabla\theta}
\providecommand{\tempGrad}{\nabla \theta}
\providecommand{\laplacianTemp}{\Delta\theta}

% Nondimensional

\providecommand{\divergenceNondim}{\nabla^{\prime}\cdot}

\providecommand{\velocityNondim}{\mathbf{v^{\prime}}}

\providecommand{\substDerivVelNondim}{\frac{D\velocity^{\prime}}{dt^{\prime}}}
\providecommand{\partialTimeVelNondim}{\partial_{t^{\prime}}\velocity^{\prime}}
\providecommand{\inertTermVelNondim}{(\velocity^{\prime}\nabla^{\prime})\velocity^{\prime}}

\providecommand{\velocityPressureTensorNondim}{\partial_k^{\prime}v^{l \prime}\partial_k^{\prime}v^{l \prime}}

\providecommand{\pressGradNondim}{\nabla^{\prime} p^{\prime}}
\providecommand{\pressLaplacianNondim}{\Delta^{\prime} p^{\prime}}
\providecommand{\laplacianVelNondim}{\Delta^{\prime}\velocity^{\prime}}

\section{Equations of Boussinesq Approximation in Dimensional Form} \label{parameters}

We will consider the simplest form of equations used to model convection in the mantle. Specifically, we will refer to the simplest special case of the equations used by the ASPECT mantle convection numerical simulation software \cite[p. 13]{aspect}. 

Namely, our equations are based on the following assumptions. Firstly, linear relationship between density and temperature perturbations in the fluid is assumed. In other words, Boussinesq approximation is used for the equation of state of the fluid: 

\begin{equation} \label{density}
\rho(\theta_0 + \theta) - \rho(\theta_0) \approx \rho(\theta_0) (\left. \frac{1}{\rho(\theta_0)}\frac{\partial \rho}{\partial \theta} \right|_{\theta_0}) ~\theta = \rho_0 \beta\theta
\end{equation}

where $\rho(\theta_0) = \rho_0$ and $\theta_0$ are the reference density and temperature of the fluid, respectively, and

$$ \beta = \left. \frac{1}{\rho_0}\frac{\partial \rho}{\partial \theta} \right|_{\theta_0}, ~ [\beta] = \frac{1}{\Theta}$$

is the \emph{volumetric coefficient of thermal expansion}\footnote{Note that $\beta < 0 $ since the density of the fluid generally decreases with the increase of temperature.} that characterizes the decrease of density with temperature. Its name makes sense if we recall the relationship between the specific volume and density:

$$ v = \frac{1}{\rho} \implies \beta = \left. \frac{1}{\rho_0}\frac{\partial \rho}{\partial \theta} \right|_{\theta_0} = \left. v_0 \frac{\partial v^{-1}}{\partial \theta} \right|_{\theta_0} = \left. -\frac{v_0}{v_0^2} \frac{\partial v}{\partial \theta} \right|_{\theta_0} = \left. -\frac{1}{v_0} \frac{\partial v}{\partial \theta} \right|_{\theta_0} $$

Secondly, the viscosity $\nu$ of the fluid is also considered constant.

Thirdly, the coefficient of thermal diffusivity $\kappa$ is considered constant. Finally, it is assumed that no sources of heat are present in the fluid.

The following notation is used to describe the process of mantle convection. The symbols $L, M, T, \Theta$ represent the dimensions of length, mass, time, and temperature, respectively; $[X]$ means the dimension of a physical quantity $X$.

\begin{itemize}

\item[] $ [\mathbf{x}] = L $ is the vector describing a location of a particle of the fluid in space.
\item[] $ [\velocity] = L T^{-1} $ is the velocity field of the flow.
\item[] $ [t] = T $ stands for time.
\item[] $\surfaceNormal$ is the unit normal to the surface of the Earth.
\item[] $ [\rho] = M L^{-3} $ stands for density.
\item[] $[g] = LT^{-2} $ is the acceleration in the uniform gravitational field.
\item[] $ [\nu] = [\frac{\mu}{\rho}] = L^{2} T^{-1} $ is the kinematic viscosity of the fluid.
\item[] $ [p] = M L^{-1} T^{-2} $ is \emph{dynamic} pressure (i.e., the one due to inertial movement of fluid, which should not be confused with thermodynamic pressure).
\item[] $ [\theta] = \Theta $ stands for temperature of the fluid.
\item[] $ [\kappa] = L^{2} T^{-1} $ is the thermal diffusivity coefficient, which is a measure of the intensity of the diffusion of temperature.

\end{itemize}

Based on these assumptions and notation, the equations simplify to the following form:\footnote{Note that all the terms in the equation \ref{stokes1_boussinesq} were divided by $\rho_0$.}

\begin{equation} \label{stokes1_boussinesq}
\nu \laplacianVel = \frac{1}{\rho_0}\pressGrad - g \theta \surfaceNormal
\end{equation}
\begin{equation} \label{stokes2_boussinesq}
\divergence \velocity = 0
\end{equation}
\begin{equation} \label{heat_eq_visc_boussinesq_prandtl}
\partialTimeTemp + \inertTermTemp = \kappa \laplacianTemp  
\end{equation} 

Naturally, these equations must be supplemented by appropriate boundary and initial conditions in order to describe a concrete system.

\subsection{Physical Meaning of the Equations}

Equations \ref{stokes1_boussinesq} and \ref{stokes2_boussinesq} are known as \emph{incompressible Stokes equations}. They accurately approximate highly viscous or slow ("creeping") flows and are different from the \emph{incompressible Navier-Stokes equations} by neglecting the inertial terms $\partialTimeVel + \inertTermVel$.

More specifically, equation \ref{stokes1_boussinesq} describes the conservation of momentum in the flow. The equation \ref{stokes2_boussinesq} is the incompressibility constraint on the flow.

The equation \ref{heat_eq_visc_boussinesq_prandtl} describes propagation of heat in the fluid in terms of the spatial and temporal evolution of the temperature field. The terms $\partialTimeTemp + \inertTermTemp$ describe the transport of temperature by the flow.\footnote{An alternative name is \emph{convection-diffusion} equation. In terms of semantics, advection is preferable to convection in this context. Indeed, \emph{convection} means the flow of the fluid in response to the heat gradients present in the fluid. \emph{Advection}, however, is a term used to describe a transport of a quantity (such as temperature, concentration of a chemical, etc.) in response to the fluid motion. That is why, in this case, temperature is \emph{advected}. Of course, advection of temperature (heat) causes convection of a fluid an vice versa, so these proceses are interrelated, However, \emph{convective heat transfer} is often referred to as \emph{convection}, but, in the light of our previous remarks, it would be more appropriate to call it \emph{advection} and \emph{advective heat transport}. However, this is a terminological inconsistency rooted in history.} The term $\kappa \laplacianTemp$ describes the transport of heat by conduction or, equivalently, diffusion of temperature.

\section{Applications of the Boussinesq Approximation}

The equations of Boussinesq approximation, both in the simplest form considered here and in more complicated ones, are used extensively in the modeling of mantle convection in Earth and other planets of the solar system. \cite{mantle_conv_in_earth_and_planets} Oceanography is another important application of the Boussinesq approximation \cite{miesen}. Moreover, there are "various thermal, geophysical, astrophysical and magnetohydrodynamic problems in the framework of 'Boussinesquian fluid dynamics'" that are based on the equations discussed in this survey. \cite{boussinesq_and_his_approximation} 

Naturally, it is important to thoroughly understand the simplest equations first before proceeding to the more complicated models used in applications. Yet the list of the above mentioned scientific fields in which Boussinesq approximation is used serves as an effective illustration of the necessity to understand this approximation for today's researcher.

\section{History of the Boussinesq Approximation}

The history of scientific study of convection is well summarized in an article by A.V. Getling \cite{getling_scholarpedia}:

\begin{quote}

The role of non-uniform heating as the producer of most types of fluid motions in the Universe was first recognised in the mid-eighteenth century, nearly simultaneously by George Hadley and Mikhail Lomonosov. Well-directed studies of convection in horizontal fluid layers heated from below trace back to Benard's experiments \cite{bernard}, in which the instability mechanism was, however, not purely thermal and was closely related to the \emph{thermocapillary effect}. Lord Rayleigh \cite{rayleigh} was the first to consider a linear problem of the onset of thermal convection in a horizontal layer, and a more comprehensive analysis of this problem was given by Pellew and Southwell \cite{pellew_and_southwell}. A highly extensive survey of the linear stability problems, including investigations of the effects of rotation and magnetic field on Rayleigh-Benard convection, was presented in a classical monograph by Chandrasekhar \cite{chandrasekhar}. Subsequent studies mainly dealt with nonlinear convection regimes and related pattern-formation processes. The volume of relevant publications has grown dramatically, and a number of monographs of a more or less wide scope summarize them [a concise review of many results that refer specifically to Rayleigh-Benard convection and were obtained by the end of the 1990s can be found in Getling \cite{getling}].
 
\end{quote}

The first theoretical analysis of convection by Lord Rayleigh marks the beginning of the study of the Boussinesq approximation and the accompanying equations. The work \cite[p. 545]{boussinesq_validity} presents a brief history of this subject matter:

\begin{quote}

Although these equations are named after Boussinesq \cite{boussinesq_article}, they seem to have been first used by Oberbeck \cite{oberbeck}. The plausibility argument given by Chandrasekhar \cite{chandrasekhar} is often referenced, but the first attempt at a detailed derivation in a dynamical situation was made by Spiegel and Veronis \cite{spiegel_and_veronis}. They considered a perfect gas of constant properties and used an order of magnitude argument. Similar assumptions and methods were used by Gebhart \cite{gebhart} and Plate \cite{plate}.

Mihaljan \cite{mihaljan} used a mathematically  rigorous small parameter expansion technique to derive the Boussinesq equations. He assumed that density was a linear function of temperature only and that the other properties were constant. A generalization of this approach was presented by Malkus \cite{malkus1, malkus2} who considered a perfect gas and allowed thermal diffusivity and viscosity to vary with temperature only.
 
\end{quote}

The theoretical study of the equations of Boussinesq approximation continues and has become especially relevant for understanding the validity of the computational models of mantle convection.

\section{Boussinesq Approximation in Computational Models of Mantle Convection}

We will provide a brief description of the software projects devoted to numerical modeling of mantle convection problems that the author is familiar with. It must be kept in mind that this software generally uses more complicated equations than those considered in this survey, although the latter can be solved as a special case of the more general equations and are often used to validate the results of the computations. More specifically, the more general equations may contain additional terms such as heat sources inside the mantle; the characteristics of the fluid such as viscosity and thermal diffusivity may be non-constant. \cite[p. 13]{aspect}.

A variety of computational software projects for modeling mantle convection has been developed and supported by the Computational Infrastructure for Geodynamics (CIG): \cite{cig_mantle_convection}

\textbf{ASPECT:} Finite element parallel code to simulate problems in thermal convection in both 2D and 3D models - currently in alpha testing.

\textbf{CitcomCU}: Finite element parallel code capable of modeling thermochemical convection in a three-dimensional domain appropriate for convection within the Earth's mantle. 

\textbf{CitcomS}: Finite element code designed to solve compressible thermochemical convection problems relevant to Earth's mantle.

\textbf{ConMan}: Finite element program for the solution of the equations of incompressible, infinite-Prandtl number convection in two dimensions, originally written by Scott King, Arthur Raefsky, and Brad Hager.

\textbf{Ellipsis3D}: Three-dimensional version of the particle-in-cell finite element code Ellipsis, a solid modeling code for visco-elastoplastic materials. The particle-in-cell method combines the strengths of the Lagrangian and Eulerian formulations of mechanics while bypassing their limitations.

\textbf{HC}: Global mantle circulation solver following Hager \& O'Connell (1981) which can compute velocities, tractions, and geoid for simple density distributions and plate velocities. 

\textbf{Taras Gerya's Finite difference code}: Professor Taras Gerya \cite{gerya} provides several examples of MATLAB programs that accompany his book. His computational methodology is the finite-difference method, which makes his book valuable since this method generally receives less attention in research papers and software projects than the finite-element method.

\section{Critique of the Boussinesq Approximation}

Following the arguments presented in the work \cite{barenblatt1} by G.I. Barenblatt, we will show that there is a flaw in the dimensional analysis of these equations that makes them, in fact, inadequate for describing mantle convection. It is important to note, however, that it does not mean that these equations have no value for other purposes, such as validation of numerical algorithms.

The main deficiency of our analysis, besides assuming the characteristics of the fluid constant the fluid itself incompressible, is the assumption that "the contribution of viscous energy dissipation to the thermal balance of the fluid" is negligible \cite[p. 545]{boussinesq_validity}:

\begin{quote} 

"In principle, the contribution of viscous energy dissipation to the thermal balance of the fluid should also be taken into account. To do this, one additional parameter must be included, the mechanical equivalent of heat $J$." \cite[p. 45]{barenblatt1}.

\end{quote}


As one can see in section \ref{parameters}, this parameter was neglected in our discussion.

Through dimensional analysis, the author obtains an important nondimensional parameter \cite[p. 46]{barenblatt1}

$$ \Pi_3 = \frac{Jc}{\beta g L} $$

As G.I. Barenblatt notes, neglecting this parameter is inappropriate for modeling mantle convection since the layers of fluid representing the mantle are not thin, unlike in the original experiments with convection in thin layers of fluid by Lord Rayleigh:

\begin{quote} 

"In what follows, we shall discuss convective motion in thin layers, for which the parameter $\Pi_3$ is large $(\Pi_3 \gg 1)$, so that the effect of this parameter on the similarity conditions may be neglected ... However, \emph{when modelling convection in the Earth's mantle the parameter $\Pi_3$ is of order unity and cannot be neglected}." \cite[pp. 46-47]{barenblatt1} 

\end{quote}

Thus, based on G.I. Barenblatt's arguments, more physically adequate model of mantle convection must include generation of heat due to viscous energy dissipation\footnote{I.e. friction between layers of fluid.}. This must be kept in mind when making conclusions based on the results of computational simulations obtained by solving the equations \ref{stokes1_boussinesq} \ref{stokes2_boussinesq}, and \ref{heat_eq_visc_boussinesq_prandtl} numerically. \cite[cf. p. 772 for the discussion of the importance of viscous dissipation in mantle convection]{mantle_conv_in_earth_and_planets}

Nevertheless, as was mentioned before, this critique does not render these equations useless: at the very least, they are useful for numerical analysis and validation of computations, as well as for demonstrating the idea of nondimensionalization.

\section{Nondimensionalization and Choice of Scales}

Given $9$ parameters describing mantle convection in our simplified model \\ ($\mathbf{x}, \velocity, \rho, \nu, \kappa, g, p, \theta, t$) and $4$ units of measurement ($L, M, T, \Theta$), by Buckingham's $\Pi$-theorem we can form $9 - 4 = 5$ nondimensional parameters describing the system. \cite{barenblatt1} %(cf. also \nameref{appendix_pi})

In order to nondimensionalize the equations \ref{stokes1_boussinesq} \ref{stokes2_boussinesq}, and \ref{heat_eq_visc_boussinesq_prandtl}, we choose characteristic scales for the length $L$, time $T$, velocity $V$, pressure $P$, and temperature $\Theta$ and perform a change of variables such that these quantities are now measured against these characteristic scales:

$$ \mathbf{x^{\prime}} = \frac{\mathbf{x}}{L}, t^{\prime} = \frac{t}{T},~ \velocityNondim = \frac{\velocity}{V},~ ~ p^{\prime} = \frac{p}{P}, ~\theta^{\prime} = \frac{\theta}{\Theta} $$  

or, equivalently,

$$ \mathbf{x} = L \mathbf{x^{\prime}}, t = T t^{\prime},~ \velocity = V \velocityNondim,~ p = P p^{\prime}, ~ \theta = \Theta \theta^{\prime} $$ 

Substituting these new variables in the equations \ref{stokes1_boussinesq} \ref{stokes2_boussinesq}, and \ref{heat_eq_visc_boussinesq_prandtl} and dropping the primes for convenience, we obtain the following system of equations:

\begin{equation}
\frac{\nu V}{L^2} \laplacianVel = \frac{P}{\rho_0 L}\pressGrad - g \Theta \theta \surfaceNormal
\end{equation}
\begin{equation}
\frac{V}{L} \divergence \velocity = 0
\end{equation}
\begin{equation}
\frac{\Theta}{T} \partialTimeTemp + \frac{\Theta V}{L} \inertTermTemp = \frac{\kappa \Theta}{L^2} \laplacianTemp  
\end{equation} 

or, after dividing by the terms both sides of the equations by the appropriate terms such that the coefficients in front of the leftmost terms become $1$, we obtain

\begin{equation} \label{stokes1_pre_nondim}
\laplacianVel = \frac{P L}{\rho_0 \nu V}\pressGrad - \frac{g \Theta L^2}{\nu V} \theta \surfaceNormal
\end{equation}
\begin{equation} \label{stokes2_pre_nondim}
\divergence \velocity = 0
\end{equation}
\begin{equation} \label{heat_eq_pre_nondim}
\partialTimeTemp + \frac{VT}{L} \inertTermTemp = \frac{\kappa T}{L^2} \laplacianTemp  
\end{equation} 

The next problem in nondimensionalizing these equations is choosing the appropriate characteristic scales for the length $L$, velocity $V$, pressure $P$, time $T$, and temperature $\Theta$ that adequately reflect the physical processes in the system. 

\subsection{Characteristic Length Scale}

The characteristic length scale $L$ is chosen based on the size of the physical domain where the flow takes place. For example, in case of a flow past a cylinder, it may be the radius of a cylinder. In case of convection in a box, it may be the length of a side of the box.

\subsection{Characteristic Temperature Scale}

The characteristic temperature scale $\Theta$ is typically defined as either the average temperature $\Theta = \theta_{avg}$ in the fluid, or as the maximum temperature difference in the fluid $\Theta = \max \theta - \min \theta$.

\subsection{Characteristic Time Scale}

There are two natural choices of scales in the context of highly viscous flows that are characteristic of mantle convection. \cite[pp. 19-20]{getling}

The first choice is based on the viscosity time scale:

$$ T_{viscosity} = \frac{L^2}{\nu} $$

The second choice is based on the heat conduction time scale: 

$$ T_{conduction} = \frac{L^2}{\kappa} $$

In order to choose the appropriate scale for the context of mantle convection, we observe that their ratio is equal to the Prandtl number:

$$ \frac{T_{conduction}}{T_{viscosity}} = \frac{\nu}{\kappa} = Pr $$ 

A commonly used approximation for the dynamics of Earth's mantle is the limit of infinite viscosity, which is equivalent to the limit of infinite Prandtl number since thermal conductivity is usually fixed or bounded: \cite[p. 267]{mantle_conv_in_earth_and_planets}

$$ \nu \longrightarrow \infty \Longleftrightarrow Pr \longrightarrow \infty$$

Physically, this limit can be interpreted as

$$ T_{viscosity} \ll T_{conduction} $$

or 

$$ T_{viscosity} \longrightarrow 0 $$

The interpretation of this limit is that information due to viscous interactions propagates infinitely fast. But infinitely fast processes cannot serve as a measure of real physical processes because they are infinitely slow relative to infinitely fast processes. Thus, the viscosity time scale becomes singular in the common mantle convection approximation, which makes it an inappropriate choice for the time scale.

Therefore, the only choice left is the time scale associated with heat conduction,

$$ T = T_{conduction} = \frac{L^2}{\kappa} $$

Another argument supporting this choice of scale is the fact that heat conduction takes place regardless of the macroscopic flow of the fluid. Since initially the fluid is at rest in typical problems of mantle convection, measurable viscous dissipation processes cannot start until the fluid has been set in an adequately intense macroscopic flow. At the same time, heat conduction starts immediately because a gradient of temperature is always present in the mantle.

\subsection{Characteristic Velocity Scale}

In typical problems of convection, the fluid is initially at rest and the researchers are interested in modeling the development of convection from the state of equilibrium. Therefore, there is no natural characteristic velocity in such context since in the early phase of convection the average velocity is close to zero but may be rather large in the developed phase. 

Therefore, it is plausible to define characteristic velocity simply as a ratio of characteristic length and time:

$$V = \frac{L}{T} = \frac{L}{\frac{L^2}{\kappa}} = \frac{\kappa}{L} $$

\subsection{Characteristic Pressure Scale}

An appropriate pressure scale for low Reynolds number flows (highly viscous or "creeping flows") is based on an estimate of viscous shear stress:

$$ \tau = \frac{\rho \nu V}{L}$$

Following \cite[pp. 433-434]{leal}, we shall verify that this choice of scale is appropriate by comparing it with another possible choice of scale based on characteristic kinetic energy density of the flow.

There are two physical choices for nondimensionalizing the pressure:

$$ P_1 = \tau = \frac{\rho \nu V}{L} = \frac{\rho \nu \kappa}{L^2} $$

based on the characteristic viscous shear stress $\tau$ in the fluid, and

$$ P_2 = KE = \rho V^2 = \frac{\rho \kappa^2}{L^2} $$ 

which is based on the characteristic kinetic energy density $KE$ of the flow.

Their ratio is equal to the Prandtl number :

$$ \frac{P_1}{P_2} = \frac{\tau}{KE} = \frac{\frac{\rho \nu \kappa}{L^2}}{\frac{\rho \kappa^2}{L^2}} = \frac{\frac{\nu}{L^2}}{\frac{\kappa}{L^2}} = \frac{T_{conduction}}{T_{viscosity}} = \frac{\nu}{\kappa} = Pr \Longleftrightarrow P_2 = \frac{1}{Pr} P_1 $$

A commonly used approximation for the dynamics of Earth's mantle is the limit of infinite viscosity, which is equivalent to the limit of infinite Prandtl number since thermal conductivity is usually fixed or bounded: \cite[p. 267]{mantle_conv_in_earth_and_planets}

$$ \nu \longrightarrow \infty \Longleftrightarrow Pr \longrightarrow \infty \Longrightarrow P_2 \longrightarrow 0 $$

That means that the viscous shear stress dominates the pressure balance in highly viscous flows and that the kinetic energy of such flows is negligible.

In addition, since the scale associated with $P_2$ vanishes, it implies that the gradient of pressure term $\pressGrad$ in the equation (\ref{stokes1_pre_nondim}) vanishes as well. This contradicts the physical fact that pressure influences highly viscous ("creeping") flows.\cite[pp. 433-434]{leal}

Thus, the appropriate choice of characteristic pressure scale is the one associated with viscous shear stress:

$$ P = P_1 = \tau = \frac{\rho \nu \kappa}{L^2}$$

\section{Nondimensional Form of the Equations of Boussinesq Approximation}

Substituting the scales discussed above into the equations \ref{stokes1_pre_nondim}, \ref{stokes2_pre_nondim}, and \ref{heat_eq_pre_nondim}, we obtain the following:

\begin{equation} \label{stokes1_nondim_no_parameters}
\laplacianVel = \pressGrad - \frac{g \Theta L^3}{\nu \kappa} \theta \surfaceNormal
\end{equation}
\begin{equation} \label{stokes2_nondim_no_parameters}
\divergence \velocity = 0
\end{equation}
\begin{equation} \label{heat_eq_nondim_no_parameters}
\partialTimeTemp + \inertTermTemp = \laplacianTemp  
\end{equation} 

Considering the definition of Rayleigh number 

$$ Ra = \frac{g\Theta \beta L^3}{\nu\kappa} $$

we obtain the following system of equations in nondimensional form:

\begin{equation} \label{NS1_viscNondim_boussinesq_prandtl}
\laplacianVel = \pressGrad - Ra ~\theta \surfaceNormal
\end{equation}
\begin{equation} \label{NS2_viscNondim_boussinesq_prandtl}
\divergence \velocity = 0
\end{equation}
\begin{equation} \label{heat_eq_viscNondim_boussinesq_prandtl}
\partialTimeTemp + \inertTermTemp = \laplacianTemp  
\end{equation} 

As we can see, the only governing parameter of the system is the Rayleigh number, which has made this approximation widely used for numerical experiments and validation of numerical algorithms for solving mantle convection equations.

\subsection{Physical Meaning of the Rayleigh Number}

According to G.I. Barenblatt, "the parameter $Ra$, the Rayleigh number, is named after the great English physicist Rayleigh who was the first to study the onset of convection in a horizontal layer theoretically," \cite[p. 47]{barenblatt1} and is a measure of the influence of temperature gradients on the mechanical motion of the fluid. 

Since

$$ Ra = \frac{g\Theta \beta L^3}{\nu\kappa} = \frac{g\Theta \beta L T}{\nu} = \frac{g\Theta \beta T}{\frac{\nu}{L}} = \frac{\rho V_{buoyancy}}{\rho V_{viscosity}} $$

it can be interpreted as a ratio of momentum gained due to the buoyant force, resulting from thermal expansion of the fluid, and the momentum dissipated by the viscosity. 

Another interpretation of Rayleigh number is as a ratio of the work of buoyant force and viscous shear stress:

$$ Ra = \frac{g\Theta \beta L^3}{\nu\kappa} = \frac{\rho g\Theta \beta L}{\frac{\rho \nu \kappa}{L^2}} = \frac{Work_{buoyancy}}{Work_{stress}} $$

In other words, Rayleigh number measures how far from the equilibrium a system is, or is a measure of the instability of the system. Indeed, while the buoyant force destabilizes the motion, viscosity stabilizes it through energy dissipation.