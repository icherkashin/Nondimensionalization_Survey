\providecommand{\norm}[1]{\lVert#1 \rVert}
\providecommand{\R}{\begin{pmatrix} R \\ 0 \end{pmatrix}}
\providecommand{\Q}{\begin{pmatrix} Q_1^{T} \\ Q_2^{T} \end{pmatrix}}
\providecommand{\SVD}{\begin{pmatrix} \Sigma \\ 0 \end{pmatrix}}
\providecommand{\SVDr}{\begin{pmatrix} \Sigma_1 & 0 \\ 0 & 0 \end{pmatrix}}
\providecommand{\V}{\begin{pmatrix} V_1^{T} \\ V_2^{T} \end{pmatrix}}
\providecommand{\U}{\begin{pmatrix} U_1  U_2 \end{pmatrix}}
\providecommand{\Vr}{\begin{pmatrix} v_1^{T} \\ \vdots \\ v_r^{T} \end{pmatrix}}
\providecommand{\Vn}{\begin{pmatrix} v_1^{T} \\ \vdots \\ v_n^{T} \end{pmatrix}}
\providecommand{\Vni}{\begin{pmatrix} v_{1i} \dots v_{ni}\end{pmatrix}}
\providecommand{\B}{\begin{pmatrix} b_1^{T} \\ \vdots \\ b_n^{T} \end{pmatrix}}

% Dimensional

\providecommand{\divergence}{\nabla\cdot}

\providecommand{\velocity}{\mathbf{v}}
\providecommand{\substDerivVel}{\frac{D\velocity}{dt}}
\providecommand{\partialTimeVel}{\partial_t\velocity}
\providecommand{\inertTermVel}{(\velocity\nabla)\velocity}

\providecommand{\velocityPressureTensor}{\partial_kv^l\partial_kv^l}

\providecommand{\surfaceNormal}{\mathbf{e}_z}
\providecommand{\pressGrad}{\nabla p}
\providecommand{\pressLaplacian}{\Delta p}
\providecommand{\laplacianVel}{\Delta\velocity}


\providecommand{\partialTimeTemp}{\partial_t\theta}
\providecommand{\inertTermTemp}{(\velocity\nabla)\theta}
\providecommand{\tempGrad}{\nabla \theta}
\providecommand{\laplacianTemp}{\Delta\theta}

% Nondimensional

\providecommand{\divergenceNondim}{\nabla^{\prime}\cdot}

\providecommand{\velocityNondim}{\mathbf{v^{\prime}}}

\providecommand{\substDerivVelNondim}{\frac{D\velocity^{\prime}}{dt^{\prime}}}
\providecommand{\partialTimeVelNondim}{\partial_{t^{\prime}}\velocity^{\prime}}
\providecommand{\inertTermVelNondim}{(\velocity^{\prime}\nabla^{\prime})\velocity^{\prime}}

\providecommand{\velocityPressureTensorNondim}{\partial_k^{\prime}v^{l \prime}\partial_k^{\prime}v^{l \prime}}

\providecommand{\pressGradNondim}{\nabla^{\prime} p^{\prime}}
\providecommand{\pressLaplacianNondim}{\Delta^{\prime} p^{\prime}}
\providecommand{\laplacianVelNondim}{\Delta^{\prime}\velocity^{\prime}}

\section{Introduction}

The derivation of the equations of the Boussinesq approximation, which describe convective processes, is presented. These equations are nondimensionalized and the physical meaning of the resulting nondimensional parameters is discussed.
 
\section{Derivation of Boussinesq Approximation}

Boussinesq approximation to the equations of convection, which consist of Navier-Stokes equations (cf. chapter \ref{navier_stokes}) and heat equation (cf. chapter \ref{heat}), is based on an assumption that the fluid is incompressible. In addition, the density of the fluid is assumed not to depend on thermodynamic pressure. Furthermore, the temperature variations in the medium are deemed small enough as to accept a linear approximation to the change of density with temperature. 

It must be noted that the variation of density with temperature is only taken into account in the momentum equations while density is constant in other equations. This speaks of the limitations of the approximation: it is only valid for the ``weak convection'' in which the perturbations of density due to temperature gradients are small. Otherwise, the variation of density with temperature must be taken into account in all of the equations, leading to a system different from the one considered here.

The derivations presented below follow the works \cite{convective_stability, boussinesq_validity}.

\subsection{Goal of Boussinesq Approximation}

Boussinesq approximation results in a system of equations that describe the evolution of the \emph{perturbations} of the parameters of the system from the initial reference parameters. For example, the heat transfer equation, in the context of Boussinesq approximation, describes the evolution of the perturbation of temperature $\theta^{\prime} = \theta - \theta_0$ from the reference temperature $\theta_0$. 

Thus, after performing the computations, it is important to remember to add the reference parameters to the computed values in order to obtain the actual information about the system: e.g. $\theta = \theta_0 + \theta^{\prime}$ 

For notational convenience, we shall drop the primes in the exposition that follows, keeping in mind that, in this context, $\theta$ describes the \emph{perturbations} of temperature, not the actual value of temperature.

\subsection{Equation of State}

The equation of state that is of interest in the context of Boussinesq approximation is density as a function of temperature and pressure:

$$ \rho = \rho(\theta, p) $$ 

Neglecting the dependence of density on pressure, and considering only linear terms, the equation of state for density for the perturbations of temperature can be written as

\begin{equation} \label{density}
\rho(\theta_0 + \theta) \approx \rho(\theta_0)(1 - \left. \frac{\theta}{\rho(\theta_0)}\frac{\partial \rho}{\partial \theta} \right|_{\theta_0}) = \rho_0(1 - \beta\theta), ~ \rho(\theta_0) = \rho_0
\end{equation}

where 

$$ \beta = \left. \frac{1}{\rho_0}\frac{\partial \rho}{\partial \theta} \right|_{\theta_0}, ~ [\beta] = \frac{1}{\Theta}$$

is the \emph{volumetric coefficient of thermal expansion} that characterizes the decrease of density with temperature. Its name makes sense if we recall the relationship between the specific volume and density:

$$ v = \frac{1}{\rho} \implies \beta = \left. \frac{1}{\rho_0}\frac{\partial \rho}{\partial \theta} \right|_{\theta_0} = \left. v_0 \frac{\partial v^{-1}}{\partial \theta} \right|_{\theta_0} = \left. -\frac{v_0}{v_0^2} \frac{\partial v}{\partial \theta} \right|_{\theta_0} = \left. -\frac{1}{v_0} \frac{\partial v}{\partial \theta} \right|_{\theta_0} $$

\subsection{Temperature Advection-Diffusion Equation}

Since the temperature advection-diffusion equation in chapter \ref{heat} is linear with respect to $\theta$ and, therefore, invariant with respect to addition of a constant to the temperature, the equation for the perturbation of temperature in the Boussinesq approximation is the same as the original equation if we, for convenience, drop the primes in $\theta^{\prime}$:

\begin{equation} \label{heat_eq}
\partialTimeTemp + \inertTermTemp = \kappa \laplacianTemp 
\end{equation}

\subsection{Navier-Stokes Equations}

The only change in the Navier-Stokes equations resulting from Boussinesq approximation concerns the dependence of density on temperature expressed by the equation (\ref{density}). Substituting (\ref{density}) into the Navier-Stokes equations in chapter \ref{navier_stokes}, we obtain

\begin{multline} \label{NS1-Poisson_boussinesq}
\rho_0(1 - \beta\theta) (\partialTimeVel + \inertTermVel) = -\rho_0 g(1 - \beta\theta)\surfaceNormal - \pressGrad + \mu\laplacianVel \\ = -\rho_0 g\surfaceNormal + \rho_0 g\beta\theta \surfaceNormal - \pressGrad + \mu \laplacianVel
\end{multline}
\begin{equation} \label{NS2-Poisson_boussinesq}
\divergence \velocity = 0
\end{equation}

The term 

$$ -\rho_0\beta\theta (\partialTimeVel + \inertTermVel)$$

on the left-hand side of (\ref{NS1-Poisson_boussinesq}) can be neglected due to the assumption $ \beta\theta \ll 1$ \cite{convective_stability}.

The mechanical equilibrium and, hence, the hydrostatic pressure is determined by the equation 

$$ \nabla p_{hydrostatic} = \rho_0 g \surfaceNormal$$

Because the hydrostatic pressure is of little interest by itself, since it describes the equilibrium case when the density is constant, it can serve as a reference pressure:

\begin{equation} \label{pressure_reference}
p^{\prime} = p - p_{hydrostatic} \Longrightarrow \nabla p^{\prime} = \nabla p - \nabla p_{hydrostatic} = \nabla p - \rho_0 g \surfaceNormal
\end{equation}

Substituting (\ref{pressure_reference}) into the right hand side of (\ref{NS1-Poisson_boussinesq}) eliminates the term $-\rho_0 g\surfaceNormal$. 

Finally, after the operations described above, (\ref{NS1-Poisson_boussinesq}) is divided by $\rho_0$. Dropping the subscript in the reference density $\rho_0$ for convenience but keeping in mind that $\rho$ still denotes the reference density, the Navier-Stokes equations of the Boussinesq approximation finally become 

\begin{equation} \label{NS1-Poisson}
\partialTimeVel + \inertTermVel = g\beta\theta \surfaceNormal - \frac{1}{\rho} \pressGrad + \nu \laplacianVel
\end{equation}
\begin{equation} \label{NS2-Poisson}
\divergence \velocity = 0
\end{equation}

\subsection{Equations of Boussinesq Approximation}

Summarizing preceding considerations, the full set of equations of Boussinesq approximation in dimensional form is the following:

\begin{equation} 
\partialTimeVel + \inertTermVel = g\beta\theta \surfaceNormal - \frac{1}{\rho} \pressGrad + \nu \laplacianVel
\end{equation}
\begin{equation} 
\divergence \velocity = 0
\end{equation}
\begin{equation} 
\partialTimeTemp + \inertTermTemp = \kappa \laplacianTemp 
\end{equation}

\section{Nondimensionalization}

Following the procedures for nondimensionalization described in chapter \ref{heat} and chapter \ref{navier_stokes}, we obtain the following dimensionless system of equations:

\begin{equation} \label{NS1-inertNondim_boussinesq_ch4}
St~ \partialTimeVel + \inertTermVel = \frac{\Theta\beta}{2Fr} \theta \surfaceNormal - \frac{1}{2 Eu}\pressGrad + \frac{1}{Re}\laplacianVel 
\end{equation}
\begin{equation} \label{NS2-inertNondim_boussinesq_ch4}
\divergence \velocity = 0
\end{equation}
\begin{equation} \label{heat_eq_inertNondim_boussinesq_ch4}
 St~ \partialTimeTemp + \inertTermTemp = \frac{1}{Pe} \laplacianTemp  
\end{equation}

In problems concerning the onset of convection, the fluid is initially at rest, so there is no natural characteristic speed of the flow $V$. Therefore, it must be derived from the characteristic length and time scales:

$$ V = \frac{L}{T} $$

Substituting this relationship into the dimensionless coefficients above, we obtain new dimensionless coefficients.

\subsubsection{Strouhal Number}

$$ St = \frac{L}{VT} = 1 $$

Physically, $St = 1$ means that the stationary and nonstationary components of the flow are in balance.


\subsubsection{Measure of Buyoant Force}

$$ \frac{\Theta\beta}{2Fr} = \frac{g \Theta \beta L}{V^2} = \frac{\rho g\Theta \beta L}{\rho V^2} = \frac{Work_{buoyancy}}{2KE} $$

Physically, the nondimensional coefficient $\frac{g \Theta \beta L}{V^2}$ is a measure of the influence of buoyant force on the velocity field. It can be interpreted as the ratio of the work done by the gravity due to buoyancy, resulting from the thermal expansion of the fluid, to the kinetic energy of the flow.

\subsubsection{Euler Number}

$$ Eu = \frac{\rho V^2}{2P} = = \frac{KE}{Work_{pressure}}$$ 

is a measure of the influence of pressure on the inertial flow. It can be interpreted as the ratio of kinetic energy density and the density of work done by the pressure.

\subsubsection{Reynolds Number}

$$ Re = \frac{LV}{\nu} = \frac{L^2}{\nu T} = \frac{T_{viscosity}}{T} $$

In this context, the Reynolds number measures the influence of viscosity processes on the overall flow.

Another way to interpret the Reynolds number is as the ratio of characteristic kinetic energy to the viscous shear stresses of the flow:

$$ KE \sim \rho V^2,~ P_{viscosity} \sim \frac{\rho \nu V}{L},~ \frac{KE}{P_{viscosity}} = \frac{LV}{\nu} = Re $$

\subsubsection{Peclet Number}

$$ Pe = PrRe = \frac{\nu}{\kappa} \frac{L^2}{\nu T} = \frac{L^2}{\kappa T} = \frac{T_{conduction}}{T} $$

In this context, Peclet number measures the influence of heat conduction processes on the overall flow.

\subsection{Nondimensional Form of Boussinesq Equations}

Considering the new nondimensional parameters,
 the equations (\ref{NS1-inertNondim_boussinesq_ch4}), (\ref{NS2-inertNondim_boussinesq_ch4}), and (\ref{heat_eq_inertNondim_boussinesq_ch4}) become
 
\begin{equation} \label{NS1-inertNondim_boussinesq_vel}
\partialTimeVel + \inertTermVel = \frac{g\Theta \beta T^2}{L} \theta \surfaceNormal - \frac{PT^2}{\rho L^2} \pressGrad + \frac{\nu T}{L^2} \laplacianVel 
\end{equation}
\begin{equation} \label{NS2-inertNondim_boussinesq_vel}
\divergence \velocity = 0
\end{equation}
\begin{equation} \label{heat_eq_inertNondim_boussinesq_vel}
\partialTimeTemp + \inertTermTemp = \frac{\kappa T}{L^2} \laplacianTemp  
\end{equation} 

Further investigation of the equations requires information about the characteristic pressure scale $P$ and the characteristic time scale $T$.
